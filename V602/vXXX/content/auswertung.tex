\section{Auswertung}
\label{sec:Auswertung}

\subsection{Überprüfung der Bragg-Bedingung}
Der Winkelbereich des Geiger-Müller-Zähhlrohrs und die gemessenene Intensität der Röntgenstrahlung werden in der Abbildung \ref{fig:Braggb}
dargestellt.
Das Maximum der Kurve wird in der Abbildung markiert. 
\begin{figure}[H]
    \centering
    \includegraphics{build/Braggbed.pdf}
    \caption{Graphische Darstellung der gemessene Impulse in Abhängigkeit der Winkel.}
    \label{fig:Braggb}
\end{figure}
\noindent Das Maximum liegt bei einem Winkel von
\begin{equation*}
    2 \theta_B = \qty{28.2}{°} \iff \theta = \qty{14.1 \pm 0.1}{°}.
\end{equation*}
Der Sollwinkel beträgt dabei $\theta_\text{B,theo} = \qty{14}{°}$.


\subsection{Das Emissionsspektrum einer Cu-Röntgenröhre}
Das gemessene Emissionsspektrum der Cu-Röntgenröhre wird in der Abbildung \ref{fig:ECu} dargestellt.
Die $K_\alpha$ und $K_\beta$ Linie sind eingezeichnet.
Der Bremsberg befindet sich vor der $K_\alpha$ Linie.
\begin{figure}[H]
    \centering
    \includegraphics{build/ECu.pdf}
    \caption{Emessionsspektrum einer Cu-Röntgenröhre.}
    \label{fig:ECu}
\end{figure}

\noindent Für eine genaure Betrachtung der beiden Linien wird das Detailspektrum aufegzeichnet.
Dieses wird in einem Kristallwinkelbereich von $\qty{17}{°}$ bis $\qty{25}{°}$ gemessen.
Jedoch werden nur die Werte zur besseren Betrachtung der Winkelbereich zwischen $\qty{19}{°}$ und $\qty{24.4}{°}$ in der Abbildung \ref{fig:Detail} dargestellt.

\begin{figure}[H]
    \centering
    \includegraphics{build/Detailsp.pdf}
    \caption{Detailspektrum der $K_\alpha$ und $K_\beta$ Kennlinien.}
    \label{fig:Detail}
\end{figure}

\noindent Es wird die minimale Wellenlänge bzw. maximale Energie bestimmt.
Dabei werden die Grenzwinkel

\begin{equation*}
    \theta_{\text{K}\alpha} = \qty{22.5}{°}
\end{equation*} 
und
\begin{equation*}
    \theta_{\text{K}\beta}  = \qty{20.2}{°}
\end{equation*}
betrachtet.

\noindent Mithilfe der Gleichung \ref{eq:GlBragg} werden die minimalen Wellenlänge berechnet.
Mit den berechneten Wellenlänge wird die Energie über 
\begin{equation}
    E = \frac{h c}{\lambda}
    \label{eq:e}
\end{equation}
in eV berechnet.
Die Ergebnisse werden in der Tabelle \ref{tab:Ergebnis1} dargestellt
\begin{table}[H]
    \centering
    \caption{Berechnete Wellenlänge und dazugehörigen Energien.}
    \label{tab:Ergebnis1}
    \begin{tabular}{c c c}
        \toprule
        $ $  & $\lambda_\text{min} / 10^{-10}\unit{\meter}$ & $E_\text{max} /10^{3}\unit{\electronvolt}$\\
        \midrule
        $K_\alpha$ & 1.541 \pm 0.006 & 8.042 \pm 0.034 \\
        $K_\beta $& 1.391 \pm 0.007 & 8.91 \pm 0.04 \\
        \bottomrule
    \end{tabular}
\end{table}

\noindent Die Halbwertsbreite der $K_\alpha$ Linie wird durch die Winkel
\begin{equation*}
    \theta_{\alpha,1} = \qty{22.25}{°}
\end{equation*}
und
\begin{equation*}
    \theta_{\alpha,2} = \qty{22.7}{°}
\end{equation*}
begrenzt.
Die der $K_\beta$ Linie durch die Winkel
\begin{equation*}
    \theta_{\beta,1} = \qty{20}{°}
\end{equation*}
und
\begin{equation*}
    \theta_{\beta,2} = \qty{20.4}{°}.
\end{equation*}
Die Differenz der Winkel wird gebildet und die Energie berechnet.
Das Auflösungsvermögen wird durch 
\begin{equation}
    A_K = \frac{E_K}{|\Delta E_K|}
\end{equation}
berechnet.
Die berechnete Energien der Halbwertsbreiten und Auflösungsvermögen werden in der Tabelle \ref{tab:Ergebnis2}
\begin{table}[H]
    \centering
    \caption{Berechnete Energie der Halbwertsbreite und das Auflösungsvermögen.}
    \label{tab:Ergebnis2}
    \begin{tabular}{c c c}
        \toprule
        $ $  & $ \Delta E /10^{2}\unit{\electronvolt} $ & $ A_K $\\
        \midrule
        $K_\alpha$ & 1.5 \pm 0.5  & 53 \pm 17 \\
        $K_\beta $ & 1.7 \pm 0.6 & 53 \pm 19 \\
        \bottomrule
    \end{tabular}
\end{table} 
Zur Berechnung der Abschirmkonstantn $\sigma_1$, $\sigma_2$ und $\sigma_3$ werden die Formel 
\begin{equation}
    \sigma_1 = z-\sqrt{\frac{E_\text{abs}}{R_\infty}}
\end{equation}
\begin{equation}
    \sigma_2 = z-2\cdot \sqrt{\frac{E_\text{abs}-E_\text{\alpha}}{R_\infty}}
\end{equation}
\begin{equation} 
    \sigma_3 = z-3\cdot \sqrt{\frac{E_\text{abs}-E_\text{\beta}}{R_\infty}} 
\end{equation} 
verwendet.
Dabei gilt für die Ordnungszahl für Kupfer $z = 29$, für $R_\infty = 13.6\unit{\electronvolt}$ und nach \cite{NIST} gilt $E_\text{abs}=8980.476\unit{\electronvolt}$.
Damit ergibt sich 
\begin{equation*}
    \sigma_1 = 3.30 
\end{equation*}
\begin{equation*}
    \sigma_2 = 12.38 \pm 0.30 
\end{equation*}
\begin{equation*}
    \sigma_3 = 22.3 \pm 2.1
\end{equation*}


\subsection{Das Absorptionsspektrum}
Die gemessenen Intensitäteten gegen den Winkel aufgetragen.
Zur Bestimmung der Absorptionsenergie wird zunächst die mittlere Intensität der K-Kante bestimmt, indem 
\begin{equation}
    I_K = I_\text{min} + \frac{I_\text{max}- I_\text{min}}{2}
\end{equation}

\noindent berechnet wird. 
Der zugehörige Winkel $\theta_k$ wird bestimmt, indem eine Hilfsgerade dey Form $y = a * x+ b$ zwischen den beiden Punkte gezogen wird, welche der mittleren Intesität am nächsten liegen.
Der Winkel wird berechnet mit 
\begin{equation}
    \theta_k = \frac{I_k - b}{a}.
\end{equation}
Die Absorptionsenergie wird über die Gleichung \ref{eq:e} berechnet.
Aus den berechneten Energien werden die Abschirmkonstanten $\sigma_k$ über die Formel \ref{eq:Abschirm} bestimmt.


\subsubsection{Zink}
Es wurde der Winkelbereich von $\qty{18}{°}$ bis $\qty{21}{°}$ gemessen. 
Die Ergebnisse sind in der Abbildung \ref{fig:AbZn} dargestellt.
\begin{figure}[H]
    \centering
    \includegraphics{build/AbsorberZn.pdf}
    \caption{Absorptionsspektrum des Zink Absorbers.}
    \label{fig:AbZn}
\end{figure}
Die mittlere Intesität beträgt
\begin{equation*}
    I_K = 452.5 Imps/s.
\end{equation*}
Aus der Hilfsgerade ergeben sich die Parameter und der Winkel
\begin{equation*}
    a = \qty{2055}{Imps/s°} ,
\end{equation*}
\begin{equation*}
    b = \qty{-81842}{Imps/s}
\end{equation*}
und
\begin{equation*}
    \theta_k = 20.02.
\end{equation*}
Aus diesen Werten lassen sich nun  Absorptionsenergie und Absorptionskonstante berechnen.
\begin{equation*}
    E_K = 8.986 \pm 0.011\unit{\kilo\electronvolt}
\end{equation*}
\begin{equation*}
    \sigma_k = 4.506 \pm 0.016
\end{equation*}


\subsubsection{Brom}
Das Absorptiosspektrum wird in der Abbildung \ref{fig:AbBr} dargestellt.
Der Winkelbereich wurde zwischen $\qty{12.5}{°}$ und $\qty{14.6}{°}$ gemessen.
Für die mittlere Intensität ergibt sich 
\begin{equation*}
    I_K = \qty{22.5}{Imps/s}.
\end{equation*}

\begin{figure}[H]
    \centering
    \includegraphics{build/AbsorberBr.pdf}
    \caption{Absorptionsspektrum des Brom Absorbers.}
    \label{fig:AbBr}
\end{figure}
\noindent Für die Parameter der Gerade und der daraus berechnete Winkel ergibt sich 
\begin{equation*}
    a = \qty{35}{Imps/s°},
\end{equation*}
\begin{equation*}
    b = \qty{-904}{Imps/s}
\end{equation*}
und
\begin{equation*}
    \theta_K = 13.23.
\end{equation*}
Für die Absorptionsenergie wird 
\begin{equation*}
    E_K = 13.442 \pm 0,025 \unit{\kilo\electronvolt}
\end{equation*}
und für die Abschirmkonstanten
\begin{equation*}
     \sigma_K = 3.881 \pm 0.029
\end{equation*}
berechnet.

\subsubsection{Gallium}
Für die Messung des Gallium Absorbers wurde der Winkelbereich von $\qty{16.5}{°}$ bis $\qty{18.2}{°}$ gemessen.
Das Absorptionsspektrum wird in Abbildung \ref{fig:AbGa} gezeigt
\begin{figure}[H]
    \centering
    \includegraphics{build/AbsorberGa.pdf}
    \caption{Absorptionsspektrum des Gallium Absorbers}
    \label{fig:AbGa}
\end{figure}
\noindent Für die mittlere Intensität ergibt sich
\begin{equation*}
    I_K = 54.5.
\end{equation*}
Aus der Geraden ergeben sich die Parameter
\begin{equation*}
    a = \qty{100}{Imps/s°}
\end{equation*}
und
\begin{equation*}
    b = \qty{-3390}{Imps/s.}
\end{equation*}
Damit ergibt sich für den Winkel 
\begin{equation*}
    \theta_K = \qty{17.22}{°},
\end{equation*}
und für die Absorptionsenergie und die Abschirmkonstanten
\begin{equation*}
     E_K = 10.394 \pm 0.015 \unit{\kilo\electronvolt},
\end{equation*}
\begin{equation*}
    \sigma_K = 3.578 \pm 0.020
\end{equation*}

\subsubsection{Strontium}
Das Absorptionsspektrum des Strontium Absorbers wird in der Abbildung \ref{fig:AbSr} dargestellt.
Dabei wurde in einem Winkelbereich zwischen $\qty{10}{°}$ und $\qty{11.8}°{}$ gemessen
Die mittlere Intensität liegt bei 
\begin{equation*}
    I_K = \qty{72}{Imps/s}.
\end{equation*}
\begin{figure}[H]
    \centering
    \includegraphics{build/AbsorberSr.pdf}    
    \caption{Absorptionsspektrum des Strontium Absorbers.}
    \label{fig:AbSr}
\end{figure}
Die Parameter der Geraden ergeben 
\begin{equation*}
    a = \qty{90}{Imps/s°}
\end{equation*}
und
\begin{equation*}
    b = \qty{-1915}{Imps/s}.
\end{equation*}
Somit ergibt sich für den Winkel
\begin{equation*}
    \theta_K = 11.03
\end{equation*}
und für die Absorptionsenergie, sowie die Abschirmkonstante
\begin{equation*}
   E_K = 16.07 \pm 0.04 \unit{\kilo\electronvolt}, 
\end{equation*}
\begin{equation*}
    \sigma_K = 4.03 \pm 0.04
\end{equation*}

\subsubsection{Moseleysche Gesetzt}
Die Absorptionsenergie $E_K$ ist laut dem Moseley-Gesetzt proportional zur Ordnungszahl $Z^2$.
Zur Bestimmung der Rydbergkonstante wird die Ordnungszahl gegen $\sqrt{E_K}$ aufgetragen.

\begin{figure}[H]
    \centering
    \includegraphics{build/Moseley.pdf}
    \caption{Ordnungszahl gegen $\sqrt{E_K}$ aufgetragen, zur Bestimmung der Rydbergkonstante.}
    \label{fig:Mos}
\end{figure}
\noindent Anhand des Verlaufs der Daten, wird eine Ausgleichsgerade verwendet.
Für die Parameter ergeben sich 
\begin{align*}
    a = 0.258 \pm 0.018 \, \frac{1}{\sqrt{eV}} \\
    b = 5.202 \pm 1.965.
\end{align*} 
Aus der Steigung $a$ lässt sich die Rydbergkonstante berechnen über
\begin{equation*}
    R_\infty = \frac{1}{a^2} = (15.1 \pm 2.1) \unit{\electronvolt}
\end{equation*}