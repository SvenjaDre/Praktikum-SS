\section{Auswertung}
\label{sec:Auswertung}

\subsection{Überprüfung der Bragg-Bedingung}
Der Winkelbereich des Geiger-Müller-Zähhlrohrs und die gemessenene Intensität der Röntgenstrahlung werden in der Abbildung \ref{fig:Braggb}
dargestellt.
Das Maximum der Kurve wird in der Abbildung markiert. 
\begin{figure}[H]
    \centering
    \includegraphics{build/Braggbed.pdf}
    \caption{Graphische Darstellung der gemessene Impulse in Abhängigkeit der Winkel.}
    \label{fig:Braggb}
\end{figure}
\noindent Das Maximum liegt bei einem Winkel von
\begin{equation*}
    2 \theta_B = \qty{28.2}{°} \iff \theta = \qty{14.1 \pm 0.1}{°}.
\end{equation*}
Der Sollwinkel beträgt dabei $\theta_\text{B,theo} = \qty{14}{°}$.


\subsection{Das Emissionsspektrum einer Cu-Röntgenröhre}
Das gemessene Emissionsspektrum der Cu-Röntgenröhre wird in der Abbildung \ref{fig:ECu} dargestellt.
Die $K_\alpha$ und $K_\beta$ Linie sind eingezeichnet.
Der Bremsberg befindet sich vor der $K_\alpha$ Linie.
\begin{figure}[H]
    \centering
    \includegraphics{build/ECu.pdf}
    \caption{Emessionsspektrum einer Cu-Röntgenröhre.}
    \label{fig:ECu}
\end{figure}
Für eine genaure Betrachtung der beiden Linien wird das Detailspektrum aufegzeichnet.
Dieses wird in einem Kristallwinkelbereich von $\qty{17}{°}$ bis $\qty{25}{°}$ gemessen.
Jedoch werden nur die Werte zur besseren Betrachtung der Winkelbereich zwischen $\qty{19}{°}$ und $\qty{24.4}{°}$ in der Abbildung \ref{fig:Detail} dargestellt.
\begin{figure}[H]
    \centering
    \includegraphics{build/Detailsp.pdf}
    \caption{Detailspektrum der $K_\alpha$ und $K_\beta$ Kennlinien.}
    \label{fig:Detail}
\end{figure}
Es wird die minimale Wellenlänge bzw. maximale Energie bestimmt.
Dabei werden die Grenzwinkel
\begin{equation*}
    \theta_\text{\alpha} = \qty{22.5}{°}
\end{equation*} 
und
\begin{equation*}
    \theta_\text{\beta} = \qty{20.2}{°}
\end{equation*}
betrachtet.
Mithilfe der Gleichung \ref{eq:GlBragg} werden die minimalen Wellenlänge berechnet.
Somit ergibt sich 
\begin{equation*}
    \lambda_\text{K \alpha}
\end{equation*}
\subsection{Das Absorptionsspektrum}

\begin{figure}[H]
    \centering
    \includegraphics{build/AbsorberZn.pdf}
    \caption{Absorptionsspektrum des Zink Absorbers.}
    \label{fig:AbZn}
\end{figure}

\begin{figure}[H]
    \centering
    \includegraphics{build/AbsorberBr.pdf}
    \caption{Absorptionsspektrum des Brom Absorbers.}
    \label{fig:AbBr}
\end{figure}

\begin{figure}[H]
    \centering
    \includegraphics{build/AbsorberGa.pdf}
    \caption{Absorptionsspektrum des Gallium Absorbers}
    \label{fig:AbGa}
\end{figure}


\begin{figure}[H]
    \centering
    \includegraphics{build/AbsorberSr.pdf}    
    \caption{Absorptionsspektrum des Strontium Absorbers.}
    \label{fig:AbSr}
\end{figure}