\section{Diskussion}
\label{sec:Diskussion}
\subsection{Überprüfung der Bragg-Bedingung}

Das gemessene Maximum bei dem Winkel $\theta_B = \qty{14.1}{°}$ besitzt eine Abweichung von 
$\qty{0.71}{\%}$ vom Sollwinkel $\theta_\text{B,theo} = \qty{14}{°}$.
Aufgrund dieser geringen Abweichung, lässt sich sagen, dass die Bragg Bedingung erfüllt ist.

\subsection{Emmissionspektrum ener Cu-Röntgenröhre}
Die Abweichung der maximialen Energie der K-Linien werden in der Tabell \ref{tab:AbwEK} dargestellt.
\begin{table}[H]
    \centering
    \caption{Abweichung der maximalen berechneten Energie der K-Linien vom Literaturwert.}
    \label{tab:AbwEK}
    \begin{tabular}{c c c c}
        \toprule
        $ $ & $E_\text{max,berechnet}/\unit{\kilo\electronvolt}$ & $E_\text{max,Theo}/\unit{\kilo\electronvolt}$ & Abweichung / $\%$\\
        \midrule
        $K_\alpha$ & 8.042 \pm 0.034 & 8.047 & 0.06 \\
        $K_\beta $ & 8.91 \pm 0.04   & 8.905 & 0.05 \\
        \bottomrule
    \end{tabular}
\end{table} 
Die Abweichungen sind sehr gering.
Die Abweichung der minimale Wellenlänge des Bremsspektrums vom berechneten Theoriewert beträgt $\qty{17}{\%}$.
Die maximale Energie wdes Bremsspektrums weicht um $\qty{21.1}{\%}$ vom Theoriewert ab.

Zunächst wird in der Tabelle \ref{tab:Abweabs} die berechneten Absorptionsenergien mit dem Literaturwert verglichen.
\begin{table}[H]
    \centering
    \caption{Abweichung der maximalen berechneten Energie der K-Linien vom Literaturwert.}
    \label{tab:Abweabs}
    \begin{tabular}{c c c c}
        \toprule
        Absorber (Ordnungszahl) & $E_\text{K,berechnet}/\unit{\kilo\electronvolt}$ & $E_\text{K,Theo}/\unit{\kilo\electronvolt}$ & Abweichung / $\%$\\
        \midrule
        Zn (30) & 8.986 \pm 0,011  & 9.65  & 6.88 \\
        Ga (31) & 10.394 \pm 0.015 & 10.37 & 0.22 \\
        Br (35) & 13.422 \pm 0.025 & 13.47 & 0.22  \\
        Sr (38) & 16.070 \pm 0.040 & 16.10 & 0.18  \\
        \bottomrule
    \end{tabular}
\end{table}
Die zu $E_K$ gehörigen Braggwinkel werden in der Tabelle \ref{tab:Abwtheta} mit dem Literaturwert verglichen.
\begin{table}[H]
    \centering
    \caption{Abweichung der maximalen berechneten Energie der K-Linien vom Literaturwert.}
    \label{tab:Abwtheta}
    \begin{tabular}{c c c c}
        \toprule
        Absorber (Ordnungszahl) & $\theta_\text{K,berechnet}/°$ & $\theta_\text{K,Theo}/°$ & Abweichung / $\%$\\
        \midrule
        Zn (30) & 20.02 & 18.6 & 7.63 \\
        Ga (31) & 17.22 & 17.3 & 0.46\\
        Br (35) & 13.23 & 13.2 & 0.22 \\
        Sr (38) & 11.03 & 11.0 & 0.27 \\
        \bottomrule
    \end{tabular}
\end{table}

\noindent Der Vergleich der Abschirmkonstante mit dem Literaturwert wird in der Tabelle \ref{tab:Abwsigs} dargestellt. 
\begin{table}[H]
    \centering
    \caption{Abweichung des Abschirmkonstante vom Literaturwert.}
    \label{tab:Abwsigs}
    \begin{tabular}{c c c c}
        \toprule
        Absorber (Ordnungszahl) & $\sigma_\text{K,berechnet}$ & $\sigma_\text{K,Theo}$ & Abweichung / $\%$\\
        \midrule
        Zn (30) & 4.506 \pm 0.016 & 3.56 & 26.57 \\
        Ga (31) & 3.578 \pm 0.02 & 3.62 & 1.16 \\
        Br (35) & 3.881 \pm 0.029 & 3.84 & 1.06  \\
        Sr (38) & 4.03 \pm 0.04 & 3.99 & 1.00  \\
        \bottomrule
    \end{tabular}
\end{table}
Allgemein lässt sich sagen, dass die größte Abweichung bei dem Zink Absorber liegt.
Bei den anderen Absorbern sind die Abweichungen gering.\\

Der Theoriewert der Rydbergkonstante lautet 
\begin{equation*}
    R_{\infty,Theo} = 13.6 \unit{\electronvolt}
\end{equation*}
und der berechnete Wert bei
\begin{equation*}
    R_{\infty,berechnet} = (15.1 \pm 2.1) \unit{\electronvolt}.
\end{equation*}
Die Abweichung beträgt $11.09\%$.
Diese größere Abweichung können dadurch entstanden sein, dass der bererchnete Wert aus den gemessenen Werten berchnet wurde,
welche eine Messunsicherheit besitzen.
