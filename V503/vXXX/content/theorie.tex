\section{Theorie}
\label{sec:Theorie}
Um die Elementarladung $e_0$ zu bestimmen wird die Öltröpfchenmethode verwendet.
Die Öltröpfchen werden dabei mit einem Zerstäuber in das elektrische Feld eines Plattenkondensators gegeben.
Die Tröpfchen erhalten ein vielfaches der Elementarladung, aufgrund der durch die Zerstäubung entstehende Reibung.
Im Plattenkondensator wirken auf dem Tröpfchen zu einem die Gewichtskraft $\vec{F}_g = mg$ und die Stokesche Reibungskraft mit
$\vec{F}_R = -6 \pi r \eta_L \vec{v}$.




\cite{sample}
