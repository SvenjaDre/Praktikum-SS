\section{Diskussion}
Die Faktor zwischen Experimental- und Literaturwert lässt sich nun durch 
\begin{equation}
    K =\frac{q_\text{exp}}{q_\text{Lit}}
\end{equation}

Der Literaturwert der Elementarladung $e$ ist $\qty{1.602e-19}{\coulomb}$\cite{PhysikTabellen}. Damit liegt der aus den genommenen Werten 
$e=\qty{7.53(0.45)e-19}{\coulomb}$ um Faktor 4,7 über dem wahren Wert. Er liegt allerdings in der gleichen Größenordnung.
Die Avogadro-Konstante liegt laut Literatur bei $N_A=\qty{6.0225e23}{}$\cite{PhysikTabellen} und ist damit 80\% größer als der Wert, der sich aus der Messung
ergibt. Dies ist allerdings schon nach der Datennahme zu erwarten gewesen, da es in jeder der Messungen der einzelnen Tröpfchen signifikante
Ausreißer in den Zeiten gab. Daher ergeben sich schon sehr ungenaue Auf- und Abstiegszeiten. Allerdings besonders die Differenz der beiden
Geschwindigkeiten ist dadurch quasi nicht bestimmbar, da durch die Abweichungen die Geschwindigkeiten teilweise gleich groß sein, aber auch um eine Größenordnung auseinander liegen können.
Dadurch ist schon das Überprüfen der Ladungsstabilität gemäß der Bedingung $2v_0=v_\text{diff}$ nicht möglich.
Nach Augenmaß trifft dieser Zusammenhang praktisch nie auf. Durch die großen Abweichungen, könnte der Zusammenhang natürlich trotzdem 
noch erfüllt sein, weswegen die Werte nicht aussortiert werden. Die Frage, die sich stellt ist, warum es während der Messung regelmäßig zu signifikanten
Ausreißern in den gemessenen Zeiten kam. Eine Erklärung besteht in der Wechselwirkung unter den Öltröpfchen, wobei dabei explizit nicht die Coulombkraft als weitere Kraft im
Kräftegleichgewicht gemeint ist, sondern wirklich die Kopplung zweier Tröpfchen zu einem gemeinsamen System. Zwei geladene Öltröpfchen
beeinflussen sich gegenseitig. Ein größeres wenig geladenes Tröpfchen kann andere Tröpfchen beim Aufstieg durch das elektrostatische Feld abbremsen,
ohne dabei bei allen Messungen einen Einfluss zu haben. Eine Analogie besteht in einer kleinen und großen Kugel mit dem gleichen Impuls. Die kleine Kugel würde sich deutlich schneller als die große bewegen.
Wenn nun die kleine auf die große trifft bewegen Sie sich als ein System mit einer gemeinsamen Masse weiter. Im Millikanfall würde eine solche Kopplung durch Umkehr der Feldrichtung aufgehoben werden. So 
können sich die Tröpfchen in machen Messungen durch andere Tropfen beim Gang durch den Kondensator anders Verhalten als in anderen Messungen.Ein weiterer Grund könnte im menschlichen Versagen bestehen. Die Tröpfchen sind sehr klein und schwer zu beobachten.
Es könnte also auch zu Verwechslungen zwischen Tröpfchen oder zu Verwechslungen in der Auf und Abstiegszeit gekommen sein, da die Darstellung der
Tröpfchen auf dem Bildschirm invertiert ist. Generell ist die Apparatur stark verunreinigt gewesen und musste mehrmals gereinigt werden. Dies könnte ebenfalls
negative Effekte auf den Versuchsablauf gehabt haben. Aus Abbildung \ref{fig:Ladung} ist auch ersichtlich, dass eher große Tröpfchen
zur Messung gewählt werden sollten, um eine klare Geschwindigkeitsdifferenz messen zu können.
\label{sec:Diskussion}
