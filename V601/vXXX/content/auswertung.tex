\section{Auswertung}
\label{sec:Auswertung}

\subsection{Der Dampfdruck und die mittlere freie Wellenlänge}

Um den Dampfdruck $p_\text{sät}$ zu berechnen wird die Formel \ref{eq:Druck} verwendet.
Für die Berechnung der mittleren freien Wellenlänge $\bar{w}$ wird die Formel \ref{eq:freieW} benutzt.
Die berechneten Werte werden in der Tabelle \ref{tab:puw} dargestellt.
Damit eine ausreichende Stoßwahrscheinlichkeit gegeben ist, sollte das Verhältnis zwischen der mittleren freien Wellenlänge und dem Abstand $a$ zwischen 1000 und 4000 liegen.
Das Verhältnis wird ebenfalls in der Tabelle \ref{tab:puw} dargestellt.
Dabei beträgt der Abstand zwischen Beschleunigunskathodde- und anode im Experiment $a = \qty{1}{cm}$ 
Da die Temperaturen ab $\qty{170}{°C}$ stärker schwanken, wird von einer Unsicherheit von $\pm\qty{2}{°C}$ ausgegangen.
Die Fehlerrechnung wurde mit $uncertainties$ durchgeführt.
\begin{table}[H]
    \centering
    \caption{Der Berechnete Dampfdruckund mittlere freie Wellenlänge. }
    \label{tab:puw}
    \begin{tabular}{c c c c c}
        \toprule
        $T/ \unit{\degreeCelsius}$ & $T/\unit{\kelvin}$ & $p_\text{sät}/\unit{\milli\bar}$ & $\bar{w}/\unit{\milli\meter}$ & $\frac{\bar{w}}{a}$\\
        \midrule
        $27.7 \pm 1$ & 300.85 & $ (6.52 \pm 0.49) \cdot 10^{-3}$   &$ (44.45 \pm 3.37)  \cdot 10^{-3}$ &$ 2.2 $\\
        $148 \pm 1 $ & 421.15 & $ 4.46 \pm 0.17             $   &$ (0.649 \pm 0.025) \cdot 10^{-3}$ &$ 1539$ \\
        $154 \pm 1 $ & 427.15 & $ 5.61 \pm 0.26             $   &$ (0.516 \pm 0.019) \cdot 10^{-3}$ &$ 1936$ \\
        $161 \pm 1 $ & 434.15 & $ 7.27 \pm 0.26             $   &$ (0.398 \pm 0.014) \cdot 10^{-3}$ &$ 2510$ \\
        $170 \pm 2 $ & 443.15 & $10.04 \pm 0.70             $   &$ (0.288 \pm 0.020) \cdot 10^{-3}$ &$ 3462$ \\
        $196 \pm 2 $ & 469.15 & $23.72 \pm 1.48             $   &$ (0.122 \pm 0.076) \cdot 10^{-3}$ &$ 8181$ \\ 
        \bottomrule
    \end{tabular}
\end{table}

\subsection{Die Differentielle Energieverteilung}
Um die Diferentielle Energeriverteilung zu bestimmen, werden die Differentialwerte auf Millimeterpapier bestimmt.
Dabei werden die aufgenommenen Werte bei einer Temperatur von $T = \qty{27.7}{\degreeCelsius}$, welche in der Abbildung \ref{fig:27} dargestellt sind, verwendet.
Die bestimmten Differentialwerte werden in der Tabelle \ref{tab:27} und in der Abbildung \ref{fig:K27} dargestellt.
\begin{table}[H]
    \centering
    \caption{Differentialwerte der Messung bei einer Temperatur von $T = \qty{27.7}{\degreeCelsius}$.}
    \label{tab:27}
    \begin{tabular}{c c c}
        \toprule
        $U_B / \unit{\volt}$ & $\Delta U_B/\unit{\volt}$ & $\Delta I/\unit{\micro\ampere}$  \\
        \midrule
        1   & 1   &  0\\
        2   & 1   &  0\\
        3   & 1   &  0.1\\
        4   & 1   &  0.12\\
        5   & 1   &  0.2\\
        6   & 1   &  0.32\\
        6.5 & 0.5 &  0.48\\
        7   & 0.5 &  0.64\\
        7.5 & 0.5 &  1.15\\
        8   & 0.5 &  1.6\\
        8.5 & 0.5 &  7.41\\
        9   & 0.5 &  0.41\\
        10  & 1   &  0 \\
        \bottomrule
    \end{tabular}
\end{table}

\begin{figure}[H]
    \centering
    \includegraphics{build/K27.pdf}
    \caption{Differentielle Energieverteilung bei einer Temperatur $T = \qty{27.7}{\degreeCelsius}$.}
    \label{fig:K27}
\end{figure}
Zur bestimmung des Kontaktpotentials $K$ wird die Formel \ref{eq:Kontakt} verwendet.
Dabei beträgt die Beschleunigungsspannung $U_B = \qty{11}{V}$.
Aus den bestimmten Differentialwerten ergibt sich die maximale Spannung $U_\text{B,max} = \qty{8.5 \pm 0.5}{V}$.
Für das Kontaktpotential ergibt sich damit 
\begin{equation*}
    K = \qty{2.5 \pm 0.5}{eV}.
\end{equation*}
Die Differentialwerte für die Messung bei einer Temperatur $T = \qty{148}{\degreeCelsius}$ werden in der Tabelle \ref{tab:148} und in der Abbildung \ref{fig:K148} 
dargestellt.
\begin{table}[H]
    \centering
    \caption{Differentialwerte der Messung bei einer Temperatur von $T = \qty{148}{\degreeCelsius}$.}
    \label{tab:148}
    \begin{tabular}{c c c}
        \toprule
        $U_B / \unit{\volt}$ & $\Delta U_B/\unit{\volt}$ & $\Delta I/\unit{\micro\ampere}$  \\
        \midrule
        0.5 & 0.5  & 3.6\\
        1   & 0.5  & 2.1\\
        1.5 & 0.5  & 2\\
        2   & 0.5  & 1.08\\
        3   & 1    & 0.36\\
        4   & 1    & 0.5\\
        5   & 1    & 0.6\\
        6   & 1    & 0.4\\
        7   & 1  & 0.3\\
        8   & 1  & 0\\
        9   & 1  & 0.03\\
        \bottomrule
    \end{tabular}
\end{table}

\begin{figure}[H]
    \centering
    \includegraphics{build/K148.pdf}
    \caption{Differentielle Energieverteilung bei einer Temperatur $T = \qty{148}{\degreeCelsius}$.}
    \label{fig:K148}
\end{figure}

\subsection{Die Franck-Hertz Kurve}
Um die Anregungsenergie für $\ce{Hg}$-Atome zu ermitteln, wurde der Abstand zwischen zwei Maxima auf dem Millimeterpapier ausgemessen.
Der Abstand $\Delta U_B$ wird von den jeweiligen Kurven bei verschiedenen Temperaturen in der Tabelle \ref{tab:DeltaUb} dargestellt.
Es wurde hierbei die Kurven \ref{fig:198}, \ref{fig:196}, \ref{fig:154}, \ref{fig:161}, \ref{fig:173} und \ref{fig:170} verwendet.

\begin{table}[H]
    \centering
    \caption{Abstand $\Delta U_B /\unit{\volt}$ der Maxima der verschiedenen Franck-Hertz-Kurven.}
    \label{tab:DeltaUb}
    \begin{tabular}{c c c c c c}
        \toprule
        $T = 198.4 \unit{\degreeCelsius} $ & $T = 196 \unit{\degreeCelsius}$ & $T = 154.1 \unit{\degreeCelsius}$ & $T = 161 \unit{\degreeCelsius}$ & $T = 173 \unit{\degreeCelsius}$ & $T = 170 \unit{\degreeCelsius}$\\
        \midrule
        4.3                                &        4.5                       &   4.8                             &    5.3                         &  4.5                            &    4.3                         \\  
        4.3                                &        4.8                       &   4.8                             &    5.6                         &  4.6                            &    4.3                          \\
        4.5                                &        4.6                       &   5.1                             &    5.3                         &  4.5                            &    4.3                          \\
        4.5                                &        5.1                       &   5.5                             &    5.8                         &  4.8                            &    4.3                          \\
        4.7                                &        5.1                       &                                   &                                &  4.8                            &    4.8                          \\
        4.5                                &        5.1                       &                                   &                                &                                 &                                 \\
        4.7                                &        5.3                       &                                   &                                &                                 &                                 \\
        4.7                                &        5.1                       &                                   &                                &                                 &                                 \\
        4.8                                &        5.1                       &                                   &                                &                                 &                                 \\
        4.3                                &                                  &                                   &                                &                                 &                                 \\
        \hline
        \multicolumn{6}{c}{Mittelwert}\\
        \midrule
        4.52 \pm 0.17 & 4.96 \pm 0.25 & 5.05 \pm 0.28 & 5.50 \pm 0.21 & 4.64 \pm 0.13 & 4.40 \pm 0.20\\
        \bottomrule
    \end{tabular}
\end{table}
\noindent Dies wird nun noch einmal gemittelt, wodurch sich der Abstand
\begin{equation*}
    \Delta U_B = \qty{4.8 \pm 0.2}{V}
\end{equation*}
ergibt.
Die Anregungsenergie beträgt somit
\begin{equation*}
    E = \qty{4.8 \pm 0.2}{eV}.
\end{equation*}
Für die Berechnung der Wellenlänge wird die Formel \ref{eq:WellenL} verwendet.
Die Wellenlänge beträgt somit
\begin{equation*}
    \lambda = \qty{257.96 \pm 0.11}{nm}
\end{equation*}
