\section{Diskussion}
\label{sec:Diskussion}

Es ist zu erkennen, dass der Dampfdruck mit steigender Temperatur ansteigt, wie es auch der Verlauf der Dampfdruckkurve aus Abbildung \ref{fig:Dampf} beschreibt.
Die mittlere freie Wellenlänge nimmt mit steigendem Dampfdruck ab, aufgrund dessen, dass bei höheren Temperaturen
die Stoßwahrscheinlichkeit zunimmt.
Bei einer Temperatur von $T = \qty{27.7}{\degreeCelsius}$ ist zu erkennen, dass die mittlere freie Wellenlänge $\bar{w}$
nicht um $1000$ bis $4000$ kleiner als der Abstand $a$ ist.
Damit sind bei dieser Temperaturen nicht ausreichende Stoßwahrscheinlichkeiten gegeben.
Bei einer Temperatur $T = \qty{196}{\degreeCelsius}$ ist dieser Faktor deutlich überschritten.
Die Stoßwahrscheinlichkeit ist somit sehr groß.\\


\noindent Die im Versuch bestimmte Anregungsenergie 
\begin{equation*}
    E = \qty{4.8 \pm 0.2}{eV}
\end{equation*}
weicht mit $2.04 \%$ vom Literaturwert ab.
Dieser liegt bei 
\begin{equation*}
    E_\text{lit} = \qty{4.9}{eV}
\end{equation*}
\cite{EAnregung}.
Die Wellenlänge $\lambda = \qty{257.96}{nm}$ liegt im Ultraviolettenbereich und ist somit für das Auge nicht sichtbar.
Der Literaturwert der Wellenlänge beträgt $\lambda_\text{lit} = \qty{253}{nm}$.
Damit weicht der im Experiment berechnet Wert um $1.96 \%$ vom Literaturwert ab.
Da ein Elektron eine Masse von $5.48 \cdot 10^{-3}$ u besitzt und ein $\ce{Hg}$ Atom eine von $200.59$ u,
ist zu sagen, dass beim elastischen Stoß der Energieverlust des Elektrons sehr gering ist.
Somit muss der Energieverlust bei der Auswertung nicht berücksichtigt werden.\\

\noindent Bei den Unsicherheiten handelt sich um systematische Fehler, die durch den verwendeten XY-Schreiber unter anderem
entstanden sind, da dieser bei jeder Messung neu kalibriert werden muss.
Beispielsweise ist bei der Messung in Abbildung \ref{fig:196} deutlich zu erkennen, dass durch falsches kalibrieren der Y-Achse,
die Maxima bei höheren Beschleunigungsspannungen flacher sind, als die vorherigen Maxima.
Des weiteren entstehen Unsicherheiten, durch das Ablesen der Messwerte vom Millimeterpapier.
Des weiteren muss bedacht werden, dass die Temperatur nicht genau einstellbar ist und nur durch einen Regler ungefähr eingestellt werden kann.
Die Temperatur mag allerdings sowohl während der Messung aber vor allem auch zwischen 2 Vergleichsmessungen schwanken.





