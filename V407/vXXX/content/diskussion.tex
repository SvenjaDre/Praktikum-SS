\section{Diskussion}
\label{sec:Diskussion}
Der Literaturwert des Brechungsindex von Silicium lautet 
\begin{equation*}
    n_\text{Literatur} = 3.88.   
\end{equation*}
\cite{n_Si}
In der Tabelle \ref{tab:abw} sind die jeweiligen Abweichungen vom Literaturwert aufgezeigt.
\begin{table}
    \centering
    \caption{Abweichungen der berechneten Brechunginidezes vom Literaturwert}
    \label{tab:abw}
    \begin{tabular}{c c c}
        \toprule
        $n$ & Ergebniss &Abweichung/$\%$ \\
        \midrule
        $\overline{n}_\perp$ & 3.65 \pm 0.59 & 5.92 \\
        $\overline{n}_\parallel$ & 6.20 \pm 5.98 & 59.79 \\
        $n_{\alpha_p}$  & 2.74 \pm 0.14 & 29.38 \\
        \bottomrule
    \end{tabular}
\end{table}
Es ist zu sagen, dass bei der Messung mit parallel polarisiertemLicht die größte Unsicherheit auftreten.
Dies ist auch anhand der Theoriekurve zu sehen, welche stark von den Messwerten abweicht.
Die Unsicherheiten könnten unteranderem entstanden sein, dass die Intensität bei den letzten gemessenen Winkeln ungenau sind,
da der Laserstrahl nicht mehr genau am Spiegel reflektiert werden konnte.
Zudem kann es zu Messungenauigkeiten durch das manuelle einstellen des Goniometers.
Desweiteren entstehen Messungenauigkeiten, beim Messen, da der reflektierte Laserstrahl den Detektor genau mittig treffen muss um die maximale Intesität zu messen.