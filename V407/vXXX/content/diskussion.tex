\section{Diskussion}
\label{sec:Diskussion}
Der Literaturwert des Brechungsindex von Silicium lautet 
\begin{equation*}
    n_\text{Literatur} = 3.88.   
\end{equation*}
\cite{n_Si}
In der Tabelle \ref{tab:abw} sind die jeweiligen Abweichungen vom Literaturwert aufgezeigt.
\begin{table}[H]
    \centering
    \caption{Abweichungen der berechneten Brechunginidezes vom Literaturwert}
    \label{tab:abw}
    \begin{tabular}{c c c}
        \toprule
        $n$ & Ergebniss &Abweichung/$\%$ \\
        \midrule
        $\overline{n}_\perp$ & 3.65 \pm 0.59 & 5.92 \\
        $\overline{n}_\parallel$ & 6.20 \pm 5.98 & 59.79 \\
        $n_{\alpha_p}$  & 2.74 \pm 0.14 & 29.38 \\
        \bottomrule
    \end{tabular}
\end{table}
\noindent Es ist zu sagen, dass bei der Messung mit parallel polarisiertem Licht die größte Unsicherheit auftreten.
Dies ist auch anhand der Theoriekurve zu sehen, welche stark von den Messwerten abweicht.
Die Unsicherheiten könnten unter anderem entstanden sein, dass die Intensität bei den letzten gemessenen Winkeln ungenau sind,
da der Laserstrahl nicht mehr genau am Spiegel reflektiert werden konnte.
Zudem kann es zu Messungenauigkeiten durch das manuelle einstellen des Goniometers.
Desweiteren entstehen Messungenauigkeiten, beim Messen, da der reflektierte Laserstrahl den Detektor genau mittig treffen muss um die maximale Intesität zu messen.
Der Theoriewert liegt bei parallel und senkrecht polarisiertem Lichts innerhalb des Fehlerintervalls.
Dadurch lässt sich auf statistische Fehler schließen.
Jedoch liegt der Theoriewert beim Brechungsindex, welcher über den Brewsterwinkelbestimmt wurde, außerhalb des Fehlerintervalls.
Somit sind die Fehler bei dieser Messung systematische.
Da beim Brewsterwinkel das Licht nicht mehr reflektiert werden sollte, sollte keine Intensität mehr gemessen werden.
Dies ist bei diesem Versuch nicht der Fall. Es wurde ein Minimum von $8 \unit{\micro\ampere}$ gemessen.
Der Versuch wurde zwar im Dunkeln durchgeführt, allerdings konnte der Raum nicht komplett abgedunkelt werden.
Somit traf immer noch etwas Licht auf den Detektor. 
Um diesen Fehler zu verringern, sollte Beispielsweise die Tür geschlossen bleiben.
Desweiteren ist eine mögliche Fehlerquelle des Aufbaus, dass durch das einstellen des Winkels durch die Drehscheibe.
Die untere Scheibe musste festgehalten werden, damit diese sich nicht bewegt.
Jedoch kann es dadurch immer noch zu kleineren verschiebung kommen, die die gemessenen Werte beeinflusst.
Damit dies nicht geschieht, müsste die untere Scheibe stabiler befestigt werden, damit diese sich nicht mehr so schnell bewegen lässt.
Zusammenfassend lässt sich sagen, dass die Bestimmung des Brechungsindexes über den Brewsterwinkel durch die Systematischen Fehler nicht gültig ist.
Da die bestimmten Brechungsindexe über die Fresnelformeln trotz großer statistischer Unsicherheiten innerhalb der Fehlerintervalle liegt, ist das Ergebniss gültig.    