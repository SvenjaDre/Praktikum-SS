\section{Theorie}
\label{sec:Theorie}
Der für Menschen hörbare Frequenzbereich liegt zwischen ca. $\qty{16}{Hz}$ bis $\qty{20}{kHz}$.
Unterhalb dieses Frequenzbereich wird als Infraschall bezeichnet.
Der Frequenzbereich von $\qty{20}{kHz}$ bis $\qty{1}{GHz}$ wird Ultraschall genannt.
Bei Frequenzen über $\qty{1}{GHz}$ wird als Hyperschall bezeichnet.

Bei der Doppler-Sonographie wird der sogenannte Doppler-Effekt verwendet.
Der Doppler-Effekt beschreibt die Frequenzänderung, wenn sich die Schallquelle und ein Beobachter relativ zueinander bewegen.
Es lassen sich vier verschiedene Fälle betracheten.
Beim ersten Fall, befindetet sich der Beobachter in Ruhe und die Schallquelle bewegt sich auf den Beobachter zu.
Die Freqenz $\nu_0$ der Quelle verschiebt sich zu einer höheren Frequenz $\nu_\text{kl}$.
Entfernt sich die Quelle vom Beobachter, verschiebt sich die Freqeunz zu einer kleineren Frequenz $\nu_\text{gr}$.
Die verschobenen Frequenzen lassen über 
\begin{equation}
    \nu_\text{kl/gr} = \frac{\nu_0}{1 \mp \frac{v}{c}}
\end{equation}
berechnen.
Dabei beschreibt v die Geschwindigkeit der Quelle.
\cite{sample}
