\section{Durchführung}
\label{sec:Durchführung}
Der Versuch besteht aus einem Ultraschall-Doppler-Generator, einem Computer mit dem Programm "FlowView", eine Pumpe und einer Ultraschallsonde. 
Bei der verwendeteten Ultraschallsonde handelt es sich um eine $\qty{2}{MHz}$ Sonde.
Die Pumpe besitzt eine Maximale Pumpleistung von $\qty{7.5}{L/min}$.
Die folgenden Messungen werden nur die Strömungsröhre mit einen Innendurchmesser von $\qty{10}{mm}$ und einem Außendurchmesser $\qty{15}{mm}$.
In der Strömungsröhre befindet sich eine Flüssigkeit, welche aus Wasser, Gycerin und Glaskugeln besteht.
Ein Doppler-Prismen mit den Einschallwinkeln $\theta $ 15°, 30° und 60° wird auf die Röhre gesetzt.
Unter dem Prisma und auf die Sonde wird etwas Gel getan, damit die Ultraschallwellen nicht von der Luft absorbiert werden.
Die Sonde wird dann auf das Prisma gesetzt für die Messungen.

\subsection{Strömungsgeschwindigkeit}
Die Strömungsgeschwindigkeit wird für jeden Einschallwinkeln bestimmt.
Zunächst wird an dem Ultraschallgenerator das Sample Volume auf Large gestellt.
Die Pumpe wird angeschaltet und eine Flussgeschwindigkeit eingestellt.
Im Programm "FlowView" wird nun die maximale gemessene Freqeunz und die mittlere Frequenz angezeigt und notiert.
Aus diesen beiden Frequenzen wird die Differenz gebildet.
Der Frequenzunterscheid wird bei jedem Dopplerwinkel für fünf verschiedenen Flussgeschwindigkeiten gemessen.

\subsection{Strömungsprofil}
Bei dieser Messung wird nur an einem Einschallwinkeln von 15° gemessen.
Am Ultraschallgenerator wird das Sample Volume auf Small eingestellt.
Zusätzlich lässt sich die Messtiefe am Generator einstellen.
Die Tiefe "Depth" wird zwischen $\qty{12}{\mu s}$ und $\qty{19.5}{\mu s}$ in jeweils 0.5 Schritten eingestellt.
Es wird dabei jedes mal die Signal intensity und die Geschwindigkeit "speed" gemessen und notiert.
Diese Messung wird einmal bei einer Pumpleistung von $\qty{45}{\%}$ und bei einer von $\qty{70}{\%}$ durchgeführt.