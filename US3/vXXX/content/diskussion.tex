\section{Diskussion}
Allgemein lässt sich nicht allzu viel zur Bestimmung der Strömungsgeschwindigkeiten sagen.
Es ist erkennbar, dass bei größeren Dopplerwinkeln die gemessenen Strömungsgeschwindigkeiten abnehmen. Dies könnte daran liegen, wie schnell das Ultraschall 
eine gewisse Messtiefe in der Dopplerflüssigkeit erreicht. Wie aus dem Strömungsprofil zu erkennen sein wird, verläuft die Strömung am Rand des Rohrs deutlich langsamer als in der Mitte.
Bei großen Winkeln dringt der Schall nu sehr langsam radial in das Rohr ein und trifft wahrscheinlich eher Glaskugeln am Rand.
Der lineare Zusammenhang, der in den Plots erkennbar ist ergibt sich direkt aus der Berechnung von $v$ nach Gleichung \eqref{eq:deltaf}. Da dort zwischen gemessenem und
berechnetem Wert ein linearer Zusammenhang besteht.
Ansonsten lässt sich noch der Mittelwert über die Dopplerwinkel bilden und mit einer abgeschätzten Strömungsgeschwindigkeit aus 
der Fließgeschwindigkeit gemäß $v_0=\frac{v_{\text{Fl}}}{\pi \cdot R^2}$ vergleichen, wobei $R$ der halbe Innendurchmesser des Rohrs ist.
Die Strömungsgeschwindigkeiten steigen erwartungsgemäß mit zunehmender Fließgeschwindigkeit an. 
\label{sec:Diskussion}
