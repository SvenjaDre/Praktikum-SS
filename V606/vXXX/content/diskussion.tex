\section{Diskussion}
\label{sec:Diskussion}

Die Theorie Werte der Suszeptibilität der verschiedenen Stoffen sind in der Tabelle \ref{tab:Theo} zu finden.
In den Tabellen \ref{tab:abeU} und \ref{tab:abwr} sind die jeweiligen Abweichungen vom Theoriewert dargestellt.
\begin{table}[H]
    \centering
    \caption{Abweichung der Suszeptibilität $\chi_U$ vom Theoriewert.}
    \label{tab:abeU}
    \begin{tabular}{c c c c}
        \toprule
        $Stoff$ & $\chi_T$ & $\chi_U$ & $Abweichung / \% $ \\
        \midrule        
        \ce{Nd2O3} & 0.0030 & 0.0054 \pm 0.0015  &    80\\ 
        \ce{Dy2O3} & 0.0254 & 0.0563 \pm 0.000000017  & 121.6  \\ 
        \ce{Gd2O3} & 0.0139 & 0.0181 \pm 0.0007  &  30.21  \\ 
        \bottomrule
    \end{tabular}
\end{table}

\begin{table}[H]
    \centering
    \caption{Abweichung der Suszeptibilität $\chi_R$ vom Theoriewert.}
    \label{tab:abwr}
    \begin{tabular}{c c c c}
        \toprule
        $Stoff$ & $\chi_T$ & $\chi_R$ & $Abweichung / \% $ \\
        \midrule        
        \ce{Nd2O3} & 0.0030 & 0.0026 \pm 0.0005   &  13.33  \\ 
        \ce{Dy2O3} & 0.0254 & 0.0235 \pm 0.00029  &  46.85 \\ 
        \ce{Gd2O3} & 0.0139 & 0.0138 \pm 0.0007   &  0.71  \\ 
        \bottomrule
    \end{tabular}
\end{table}

\noindent Es ist zu sehen das die Suszeptibilität $\chi_U$ des Stoffes \ce{Dy2O3} die größte Abweichung aufweist.
Die geringste Abweichung hat der Stoff \ce{Gd2O3} mit der Suszeptibilität $\chi_R$.
Im Allgemeinen ist zu sagen, dass die Suszeptibilitäten $\chi_U$ am größten von den jeweiligen Theoriewerten abweicht.
Die großen Abweichungen könnten durch Fehler durch das ablesen der Brückenspannungen entstehen.
Da bei beiden Methoden große Abweichungen auftreten, ist keine am besten geeignet.





