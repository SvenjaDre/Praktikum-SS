\section{Zielsetzung}
Ziel des Versuchs ist die Bestimmung der Suszeptibilität der seltenen Erden
\ce{Nd2O3},\ce{Gd2O3} und \ce{Dy2O3}.
\section{Theorie}
\label{sec:Theorie}
\subsection{Magnetismus in Materie}
Im Vakuum ergibt sich die magnetische Flussdichte $\vec{B}$ aus der magnetischen Feldstärke $\vec{H}$
durch 
\begin{equation}
    \vec{B}= \mu_0 \vec{H},
    \label{eq:BVakuum}
\end{equation}
wobei $\mu_0$ die Induktionskonstante im Vakuum ist.
Wirkt nun das Feld $\vec{H}$ auf Materie, kann diese um $\vec{M}$ magnetisiert werden und ändert
die magnetische Flussdichte in der Form
\begin{equation}
    \vec{B}= \mu_0 \vec{H}+\vec{M}.
    \label{eq:BMaterie}
\end{equation}
Die Magnetisierung $\vec{M}$ wird durch $N$ atomare magnetische Momente der Form 
$\vec{M}=N\mu_0 \vec{H}$ hervorgerufen. Da diese atomaren magnetischen Momente ebenfalls durch 
das äußere Feld $\vec{H}$ hervorgerufen wird, hängt $\vec{M}$ auch direkt vom äußeren Feld ab.
Daher kann der Zusammenhang näherungsweise linear über die Suszeptibilität $\chi$ beschrieben werden.
Es ergibt sich
\begin{equation}
    \vec{M}= \mu_0 \chi \vec{H}.
    \label{eq:MMakro}
\end{equation}
Allgemein ist die Suszeptibilität $\chi$ nicht konstant, sondern eine Funktion von $T$, der Temperatur, und von $\vec{H}$ selbst.
Diese Suszeptibilität soll im Folgenden einmal theoretisch aus der Quantenmechanik und einmal experimentell
durch die makroskopische Veränderung der Flussdichte in Spulen untersucht werden.
Dabei werden paramagnetische Materialien untersucht. Der Paramagnetischer Magnetismus,
tritt anders als der Diamagnetismus, der bei allen Atomen auf Grund von induzierten Momenten 
auftritt, nur bei Materialien auf, die schon im feldfreien Raum nicht verschwindende atomare
Drehimpulse aufweisen. Während beim Diamagnetismus das Feld gemäß des Induktionsgesetzes abgeschwächt,
richten sich beim Paramagnetismus die existierenden Momente in Richtung des äußeren magnetischen 
Felds aus und verstärken dieses im Material. Daher ist die Suszeptibilität dort größer 0.
\subsection{Quantenmechanische Berechnung der Suszeptibilität}
Nun soll die Suszeptibilität in Abhängigkeit des atomaren Gesamtdrehimpulses $\vec{J}$ bestimmt werden.
Dieser ergibt sich nach LS-Kopplung aus der Vektorsumme 
\begin{equation}
    \vec{J}= \vec{L}+\vec{S}.
    \label{eq:LS}
\end{equation}
Wobei $\vec{L}$ den Gesamtbahndrehimpuls beschreibt und sich aus den Einzeldrehimpulsen $\vec{L}=\sum\vec{l_i}$
und den Gesamtspin $\vec{S}$, der sich durch $\vec{S}=\sum \vec{s_i}$ berechnen lässt.
Die Beträge $\left|\vec{L}\right|$, $\left|\vec{S}\right|$, $\left|\vec{J}\right|$ ergeben sich aus den
entsprechenden Quantenzahlen $L$,$S$,$J$ durch 
\begin{align}\label{eq:absdreh}
    \left|\vec{L}\right|&=\sqrt{L(L+1)}\hbar\\
    \left|\vec{S}\right|&=\sqrt{S(S+1)}\hbar\\
    \left|\vec{J}\right|&=\sqrt{J(J+1)}\hbar.
\end{align}
Diese elementaren Drehimpulse lassen sich über das Bohrsche Magneton $\mu_B=\frac{e_0}{2m_0}\hbar$ und
$g_s$, dem gyromagnetischen Verhältnis des freien Elektrons, in magnetische Momente überführen durch
\begin{align}\label{eq:absmu}
    \left|\vec{\mu_L}\right|&=\mu_B\sqrt{L(L+1)}\\
    \left|\vec{\mu_S}\right|&=g_s \mu_B\sqrt{S(S+1)}.
\end{align}
Die einzelnen Komponenten des magnetischen Moments aus Spin und Bahndrehimpuls können nicht direkt gemessen werden, sondern
sie ergeben ein gemeinsamen Drehimpuls $\vec{J}$. Daher gehen nur dazu parallele Anteile gemäß
$\left|\vec{\mu_J}\right|=\left|\vec{\mu_S}\right|\cos{\alpha}+\left|\vec{\mu_L}\right|\cos{\beta}$ ein.
Unter Verwendung geometrischer Identitäten und der Annahme, dass $g_s$ ungefähr 2 ist, ergibt sich als Zusammenhang für
das magnetische Moment $\left|\vec{\mu_J}\right|$
\begin{equation}
    \left|\vec{\mu_J}\right| \approx \mu_B g_j \sqrt{J(J+1)}.
    \label{eq:muj}
\end{equation}
$g_j$ wird der Landé-Faktor genannt und kann direkt aus den Quantenzahlen des Atoms gemäß
\begin{equation}
    g_J=\frac{3J(J+1)+\{S(S+1)-L(L+1)\}}{2J(J+1)}
\end{equation}
berechnet werden.
Nun muss noch beachtet werden, dass das magnetische Moment der z-Komponente des Gesamtdrehimpuls gequantelt ist.
Es gilt $\left|\vec{\mu_{J_z}}\right| \approx \mu_B g_j m$. $m$ ist eine Orientierungsquantenzahl, die ganzzahlige Werte von -J bis J
annehmen kann. Dieser Effekt, beziehungsweise die Aufspaltung der entsprechenden Energieniveaus, wird Zeemann-Effekt genannt.
Wenn nun die Besetzungswahrscheinlichkeit der Energieniveaus gemäß der Boltzmannverteilung genutzt wird, kann das magnetische Moment $\overline{\mu}$ über
alle Orientierungen gemittelt werden. Nun kann das mittlere magnetische Moment für
große Temperaturen genähert werden. Dann ergibt sich die Suszeptibilität zu 
\begin{equation}
    \chi= \frac{\mu_0\mu_B^2g_J^2NJ(J+1)}{3kT}.
    \label{eq:chi}
\end{equation}
Daher lässt sich erkennen, dass die Suszeptibilität mit $1/T$ geht und damit dem Curieschen Gesetz für den paramagnetismus folgt.
Für Messungen bei Raumtemperatur wird demnach für eine Temperatur von $T=\qty{293.15}{\kelvin}$ angenommen.
Die Teilchenzahl pro Volumen lässt sich durch die Formel
\begin{equation}
    N=N_A\frac{\rho}{M_\text{mol}}
\end{equation}
berechnen. Dabei ist $N_A$ die Avogadro-Konstante, $\rho$ die reale Dichte des Stoffs und $M_\text{mol}$ die molare Masse des Stoffs.
\subsection{Bestimmung der Gesamtdrehimpulse von Atomen }
Das elementare magnetische Moment wird durch elementare Drehimpulse erzeugt. In 
voll besetzten Schalen heben sich die elementaren Spinmomente und Bahndrehimpulsmomente vollständig weg.
In leichten Atomen mit ausschließlicher Besetzung von niedrigen Energieniveaus, sind die paramagnetischen
Phänomene noch kaum beobachtbar, da nur kleine Bahndrehimpulse erreicht werden können.
Erst bei seltenen Erden die eine unabgeschlossene 4f-Schale besitzen können Elektronen mit
großen Bahndrehimpulsquantenzahlen bei vielen gleichgerichteten Spins aufnehmen.

\subsubsection{Besetzungsregeln}
Diese für den Paramagnetismus günstige Besetzung ergibt sich aus den Hundschen Regeln
\begin{enumerate}
    \item{ Die Spins $\vec{s_i}$ kombinieren zum maximalen Gesamtspin $\vec{S}=\sum\vec{s_i}$, der nach dem Pauli
    Prinzip möglich ist. Daher die Elektronen müssen sich in mindestens einer Quantenzahl unterscheiden.}
    \item Die Bahndrehimpulse $\vec{l_i}$ setzen sich so zusammen, dass der maximale Drehimpuls 
    $\vec{L}=\sum\vec{l_i}$, der mit dem Pauli-Prinzip und der Regel verträglich ist, entsteht.
    \item Der Gesamtdrehimpuls ist $J= \left|L-S\right|$, wenn die Schale weniger als halb und 
    $J=L+S$, wenn die Schale mehr als halb gefüllt ist.
    \label{Hundsche_Regel}
\end{enumerate}

\subsubsection{Suszeptibilität von \ce{Nd2O3}}
\subsubsection{Suszeptibilität von \ce{Gd2O3}}
\subsubsection{Suszeptibilität von \ce{Dy2O3}}
\subsection{Abgleichbedingungen der Brückenschaltung}
\cite{sample}
