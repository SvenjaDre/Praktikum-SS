\section{Diskussion}
\label{sec:Diskussion}
\subsection{Einordnung der bestimmten Austrittsarbeit und dem Verhältnis $h/e$}
Die Qualität der Messung kann Anhand des Verhältnisses vom plankschen Wirkungsquantum $h=\qty{6.62e-34}{\joule\second}$ und
der Elementarladung $e_0=\qty{1.6e-19}{\coulomb}$ abgeschätzt werden. Dabei beträgt der experimentelle Wert $\frac{h}{e_0}=\qty{-3.42(0.008)e-15}{\volt\second}$ und der theoretische Wert
\begin{equation}
    \frac{h}{e_0}_\text{Theo}=\qty{4.14e-15}{\volt\second}.
\end{equation}
Damit ist der theoretische Wert um 17,3\% größer als der experimentell bestimmte. Wobei die Abweichung gemäß
\begin{equation}
    \Delta = 1-\frac{h/e_0}{h/{e_0}_\text{Theo}}
\end{equation}
bestimmt worden ist.
Diese Abweichung lässt sich hauptsächlich dadurch begründen, dass nur 3 Spektrallinien für die Bestimmung der Ausgleichsgerade herangezogen worden sind.
Die rote Spektrallinie ist aufgrund ihrer niedrigen Intensität zu schnell in die Sättigung übergegangen, während im niedrigen Stromstärkenbereich die Auflösung des Picoamperemeters
nicht genau genug ist und somit Ströme zwischen 0 und 1pA nicht genau aufgelöst werden können. Die Grenzspannungen der 3 Spektrallinien ließen sich allerdings sehr genau linearisieren, 
wodurch ein zusätzlicher systematischer Fehler wahrscheinlich ist. Möglicherweise ist die Photozelle auch etwas zu alt und ein Teil des Kathoden- und Anodenmaterials ist in den
eigentlich evakuierten Innenraum verdampft. Die Wechselwirkung könnte alle Messungen gleichermaßen beeinflussen. Ein weiterer Aspekt, der verfälschend einwirkt, ist, dass
alle Elektronen	zusätzlich noch das Anodenpotential überwinden müssen. Dadurch ist die gemessene Grenzspannung nicht diejenige
bei der keine Elektronen die Anode erreichen, sondern nur die bei der die Elektronen die Austrittsarbeit der Anode nicht mehr aufbringen können.
Dies sollte allerdings nur eine konstante Verschiebung der Geraden aus Abbildung \ref{fig:Grenz} bewirken und somit nur die bestimmte Austrittsarbeit verfälschen.
Generell sollten mehr Spektrallinien gemessen werden, um das Verhältnis $h/e$ besser zu quantifizieren.
\subsection{Diskussion der Photostromabhängigkeit der 20V Messung}
Nun soll noch die Abhängigkeit des Photostroms von der anliegenden Spannung 
für die gelbe Spektrallinie, wie Sie in Abbildung \ref{fig:Allgelb} zu sehen ist, diskutiert werden.
Dabei ist bei hohen Bremsspannungen ein geringer negativer Strom zu erkennen.
Dies lässt sich dadurch erklären, dass das Kathodenmaterial bei 20° bereits verdampfen kann und sich so auch zuteil an der Anode anlagern kann.
Bei einer anliegenden Bremsspannung tritt der Photoeffekt in die genau gegenläufige Richtung auf. Da aber nur wenig Material im Bereich der Anode Elektronen generieren kann, sättigt der Strom
bereits bei geringen Spannungen. Bei Spannungen etwas über der Grenzspannung ist, wie in Abbildung \ref{fig:gelb} zu sehen,
der Photostrom quadratisch bei Vergrößerung der Spannung. Der Strom hört also nicht bei $U_G$ abrupt auf zu fließen, sondern wird schon bei größeren Spannungen gedämpft.
Das hängt damit zusammen, dass die Fermi-Energie nur den höchstmöglichen Energiezustand der Elektronen kennzeichnet.
Die Photonen können aber mit ihrer Energie auch durchaus Elektronen mit geringerer Energie herauslösen und diese können bereits bei deutlich geringeren Spannungen nicht mehr die Potentialbarriere 
überwinden. Bei großen Beschleunigungsspannungen erreicht der Strom dann eine Sättigung, da die Elektronen-Bildung durch die Intensität des Lichtes, also die Anzahl an eintreffenden Photonen pro Zeit,
gedeckelt ist. 