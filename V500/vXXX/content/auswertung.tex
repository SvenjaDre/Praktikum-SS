\section{Auswertung}
\label{sec:Auswertung}
\subsection{2V-Messung}
Die Messung von -2V bis 2V sollte für die rote, grüne und violette Spektrallinie mit der größten Intensität
aufgenommen werden. Dabei entspricht die rote Spektrallinie der Wellenlänge 623nm , die grüne der Wellenlänge 546nm und
die violette der Wellenlänge 435nm \cite{sample},\cite{Spektrum}. Für jede Spektrallinie 
ist der durch Einfall des entsprechendes Licht erzeugte Photostrom $I_\text{Ph}$ gemessen worden. Dabei ist die Spannung zwischen Photokathode und
Anode von -2V bis 2V variiert worden. Dabei bedeuten negative Spannungen eine Bremsspannung. Also eine Spannung die Elektronen in Richtung
Kathode beschleunigen und eine positive Spannung bedeutet eine beschleunigende Spannung in Richtung Anode. Die Messwerte sind in den Tabellen \ref{tab:2VMesswerte1} und \ref{tab:2VMesswerte2}
aufgeführt.
\begin{table}[H]
  \centering
  \caption{Messwerte für die rote, grüne und violette Spektrallinie(1).}
  \label{tab:2VMesswerte1}
  \begin{tabular}{S[table-format=1.2] S[table-format=3] S[table-format=1.2] S[table-format=4]S[table-format=1.2]S[table-format=2]}
      \toprule
      {$U_\text{Grün}$/V}&{$I_\text{GrPh}$/pA}&{$U_\text{violett}$/V}&{$I_\text{ViPh}$/pA}&{$U_\text{Rot}$/V}&{$I_\text{RotPh}$/pA}\\
      \midrule
      -2 & -20 & -2 & -25 & -2 & 0 \\
      -1,9 & -20 & -1,9 & -24 & -1,9 & 0 \\
      -1,8 & -20 & -1,8 & -24 & -1,8 & 0 \\
      -1,7 & -20 & -1,7 & -24 & -1,7 & 0 \\
      -1,6 & -19 & -1,6 & -23 & -1,6 & 0 \\
      -1,5 & -19 & -1,5 & -22 & -1,5 & 0 \\
      -1,4 & -19 & -1,4 & -20 & -1,4 & 0 \\
      -1,3 & -19 & -1,3 & -18 & -1,3 & 0 \\
      -1,2 & -18 & -1,25 & -16 & -1,2 & 0 \\
      -1,1 & -17 & -1,2 & -13 & -1,1 & 0 \\
      -1 & -16 & -1,15 & -10 & -1 & 0 \\
      -0,9 & -16 & -1,1 & -6 & -0,9 & 0 \\
      -0,8 & -14 & -1,05 & -2 & -0,8 & 0 \\
      -0,7 & -10 & -1 & 1 & -0,7 & 0 \\
      -0,65 & -6 & -0,95 & 9 & -0,6 & 0 \\
      -0,6 & -1 & -0,9 & 18 & -0,5 & 1 \\
      -0,55 & 2 & -0,85 & 28 & -0,45 & 2 \\
      -0,5 & 15 & -0,8 & 41 & -0,4 & 2 \\
      -0,45 & 29 & -0,75 & 57 & -0,35 & 3 \\
      -0,4 & 49 & -0,7 & 74 & -0,3 & 4 \\
      -0,35 & 78 & -0,65 & 95 & -0,25 & 5 \\
      -0,3 & 110 & -0,6 & 110 & -0,2 & 6 \\
      -0,25 & 140 & -0,55 & 140 & -0,15 & 7 \\
      -0,2 & 160 & -0,5 & 150 & -0,1 & 8 \\
      -0,15 & 190 & -0,45 & 190 & -0,05 & 10 \\
      \bottomrule
  \end{tabular}
\end{table}

\begin{table}[H]
  \centering
  \caption{Messwerte für die rote, grüne und violette Spektrallinie(2).}
  \label{tab:2VMesswerte2}
  \begin{tabular}{S[table-format=1.2] S[table-format=3] S[table-format=1.2] S[table-format=4]S[table-format=1.2]S[table-format=2]}
      \toprule
      {$U_\text{Grün}$/V}&{$I_\text{GrPh}$/pA}&{$U_\text{violett}$/V}&{$I_\text{ViPh}$/pA}&{$U_\text{Rot}$/V}&{$I_\text{RotPh}$/pA}\\
      \midrule
      -0,1 & 220 & -0,4 & 210 & 0 & 11 \\
      -0,05 & 250 & -0,3 & 250 & 0,05 & 12 \\
      0 & 270 & -0,2 & 300 & 0,1 & 14 \\
      0,05 & 290 & -0,1 & 340 & 0,15 & 15 \\
      0,1 & 310 & 0 & 380 & 0,2 & 16 \\
      0,15 & 330 & 0,1 & 420 & 0,25 & 18 \\
      0,2 & 350 & 0,2 & 460 & 0,3 & 19 \\
      0,3 & 390 & 0,3 & 490 & 0,35 & 20 \\
      0,4 & 420 & 0,4 & 530 & 0,4 & 22 \\
      0,5 & 460 & 0,5 & 560 & 0,5 & 24 \\
      0,6 & 480 & 0,6 & 600 & 0,6 & 26 \\
      0,7 & 520 & 0,7 & 630 & 0,7 & 29 \\
      0,8 & 550 & 0,8 & 660 & 0,8 & 31 \\
      0,9 & 570 & 0,9 & 670 & 0,9 & 34 \\
      1 & 600 & 1 & 710 & 1 & 36 \\
      1,1 & 650 & 1,1 & 750 & 1,1 & 38 \\
      1,2 & 680 & 1,2 & 790 & 1,2 & 40 \\
      1,3 & 720 & 1,3 & 830 & 1,3 & 43 \\
      1,4 & 740 & 1,4 & 860 & 1,4 & 45 \\
      1,5 & 770 & 1,5 & 910 & 1,5 & 47 \\
      1,6 & 800 & 1,6 & 960 & 1,6 & 49 \\
      1,7 & 830 & 1,7 & 1000 & 1,7 & 51 \\
      1,8 & 860 & 1,8 & 1050 & 1,8 & 54 \\
      1,9 & 900 & 1,9 & 1100 & 1,9 & 56 \\
      2 & 940 & 2 & 1150 & 2 & 58 \\
      \bottomrule
  \end{tabular}
\end{table}
\noindent Da der Photostrom in der Nähe der Grenzspannung quadratisch zur Änderung der Spannung geht, wird die Wurzel 
der Stromstärke gegen die Spannung aufgetragen. In der Nähe der Grenzspannung wird dann eine Lineare Regression mittels der Funktion
Polyfit aus der Bibliothek Numpy aus Python durchgeführt \cite{numpy}. Gesucht ist dann die Grenzspannung bei der kein Photostrom am 
Picoamperemeter ankommt. Wenn $a$ die Steigung der Ausgleichsgerade ist und $b$ der Y-Achsenabschnitt ist, ergibt sich die Grenzspannung $U_\text{Grenz}$ als Nullstelle durch 
\begin{equation}
  U_\text{Grenz}=-\frac{b}{a}.
  \label{eq:Grenz}
\end{equation}
Der Fehler der linearen Regression $\Delta a$ und $\Delta b$ pflanzt sich nach dem Gaußschen Fehlerfortpflanzungsgesetz
zu 
\begin{equation*}
  \Delta U_\text{Grenz}= U_\text{Grenz}\sqrt{\left(\frac{\Delta b}{b}\right)^2+\left(\frac{\Delta a}{a}\right)^2},
\end{equation*}
wobei der Fehler automatisch über die Bibliothek uncertainties in Python berechnet worden ist \cite{uncertainties}.
\\
\noindent Die erste Spektrallinie die ausgewertet wird, ist die rote. Die Messdaten sind in Abbildung \ref{fig:rot} aufgetragen.
Wegen der schnellen Sättigung des Stroms ist die lineare Regression nur für die ersten 3 Messpunkte mit positivem Strom durchgeführt worden.
Als Fit-Parameter ergeben sich dann
\begin{align*}
  a_\text{rot}&=\qty{9.51(0.425)}{\sqrt{\pico\ampere}\per\volt}\\
  b_\text{rot}&=\qty{5.72(0.221)}{\sqrt{\pico\ampere}}.\\
\end{align*}
Nach Gleichung \eqref{eq:Grenz} ergibt sich die Grenzspannung dann zu
\begin{equation*}
  U_\text{rot}=\qty{-0.6(0.04)}{\volt}.
\end{equation*}
\begin{figure}[H]
  \centering
  \includegraphics{content/rot.pdf}
  \caption{Messung der Stromabhängigkeit bei Variation der Brems-/Beschleunigungsspannung der roten Spektrallinie}
  \label{fig:rot}
\end{figure}

\noindent Die Messdaten für die grüne Spektrallinie wird in Abbildung \ref{fig:grün}
aufgetragen. Die Ausgleichsrechnung wird mit den ersten 7 Messpunkte mit positiver 
Stromstärke durchgeführt.
Als Fit-Parameter ergeben sich
\begin{align*}
  a_\text{grün}&=\qty{34.2(1.08)}{\sqrt{\pico\ampere}\per\volt}\\
  b_\text{grün}&=\qty{20.7(0.44)}{\sqrt{\pico\ampere}}.\\
\end{align*}
Wegen Gleichung \eqref{eq:Grenz} ergibt sich die Grenzspannung dann zu
\begin{equation*}
  U_\text{rot}=\qty{-0.604(0.023)}{\volt}.
\end{equation*}
\begin{figure}[H]
  \centering
  \includegraphics{content/grün.pdf}
  \caption{Messung der Stromabhängigkeit bei Variation der Brems-/Beschleunigungsspannung der grünen Spektrallinie}
  \label{fig:grün}
\end{figure}

\noindent In Abbildung \ref{fig:violett} ist der Photostrom der violetten Spektrallinie gegen die 
Spannung aufgetragen. Es sind die ersten 12 Messungen für Ströme größer 0 für eine 
Lineare Regression herangezogen worden. Die Fit-Parameter lauten 
\begin{align*}
  a_\text{violett}&=\qty{22(0.603)}{\sqrt{\pico\ampere}\per\volt}\\
  b_\text{violett}&=\qty{23.8(0.45)}{\sqrt{\pico\ampere}}.\\
\end{align*}
Die Grenzspannung ergibt sich dann durch \eqref{eq:Grenz} zu
\begin{equation*}
  U_\text{violett}=\qty{-1.08(0.04)}{\volt}.
\end{equation*}
\begin{figure}[H]
  \centering
  \includegraphics{content/lila.pdf}
  \caption{Messung der Stromabhängigkeit bei Variation der Brems-/Beschleunigungsspannung der violetten Spektrallinie}
  \label{fig:violett}
\end{figure}
\subsection{20V-Messung}
Nun wird noch die Spannungsabhängigkeit des Photostroms für eine Spektrallinie, die 
gelbe mit einer Wellenlänge von 578nm \cite{sample}, genauer über einem Spannungsbereich von -20V 
bis 20V gemessen. Dabei ist auch hier wieder um den Nulldurchgang des Stroms sehr genau in 0,5V Schritten gemessen worden.
Die Messwerte sind in Tabelle \ref{tab:20VMesswerte} eingetragen. Die gesamte Abhängigkeit zwischen Strom und Spannung 
ist in Abbildung \ref{fig:Allgelb} dargestellt.
\begin{table}[H]
  \centering
  \caption{Messwerte für die gelbe Spektrallinie von -20V bis 20V.}
  \label{tab:20VMesswerte}
  \begin{tabular}{S[table-format=2.1] S[table-format=2] S[table-format=1.2] S[table-format=2]S[table-format=1.2]S[table-format=3]S[table-format=1.1]S[table-format=3]S[table-format=2.1]S[table-format=4]}
      \toprule
      {$U$/V}&{$I_\text{Ph}$/pA}&{$U$/V}&{$I_\text{Ph}$/pA}&{$U$/V}&{$I_\text{Ph}$/pA}&{$U$/V}&{$I_\text{Ph}$/pA}&{$U$/V}&{$I_\text{Ph}$/pA}\\
      \midrule
      -19 & -16 & -4 & -10 & -0,25 & 48 & 2,6 & 500 & 6,2 & 800 \\
      -18,5 & -16 & -3,5 & -10 & -0,2 & 62 & 2,7 & 520 & 6,4 & 800 \\
      -18 & -15 & -3 & -10 & -0,15 & 77 & 2,8 & 540 & 6,6 & 820 \\
      -17,5 & -15 & -2,5 & -10 & -0,1 & 88 & 2,9 & 560 & 6,8 & 820 \\
      -17 & -15 & -2,4 & -9 & 0 & 100 & 3 & 570 & 7 & 840 \\
      -16,5 & -14 & -2,3 & -9 & 0,1 & 120 & 3,1 & 580 & 7,2 & 860 \\
      -16 & -14 & -2,2 & -9 & 0,2 & 140 & 3,2 & 600 & 7,4 & 880 \\
      -15,5 & -14 & -2,1 & -9 & 0,3 & 170 & 3,3 & 620 & 7,6 & 880 \\
      -15 & -14 & -2 & -9 & 0,4 & 190 & 3,4 & 620 & 7,8 & 900 \\
      -14,5 & -14 & -1,9 & -9 & 0,5 & 210 & 3,5 & 640 & 8 & 920 \\
      -14 & -14 & -1,8 & -9 & 0,6 & 230 & 3,6 & 650 & 8,2 & 920 \\
      -13,5 & -14 & -1,7 & -9 & 0,7 & 250 & 3,7 & 660 & 8,4 & 920 \\
      -13 & -14 & -1,6 & -9 & 0,8 & 260 & 3,8 & 680 & 8,6 & 920 \\
      -12,5 & -14 & -1,5 & -9 & 0,9 & 280 & 3,9 & 670 & 8,8 & 920 \\
      -12 & -13 & -1,4 & -9 & 1 & 300 & 4 & 680 & 9 & 920 \\
      -11,5 & -13 & -1,3 & -8 & 1,1 & 310 & 4,1 & 680 & 9,5 & 940 \\
      -11 & -13 & -1,2 & -8 & 1,2 & 320 & 4,2 & 680 & 10 & 960 \\
      -10,5 & -13 & -1,1 & -8 & 1,3 & 330 & 4,3 & 680 & 10,5 & 980 \\
      -10 & -13 & -1 & -8 & 1,4 & 340 & 4,4 & 680 & 11 & 1000 \\
      -9,5 & -12 & -0,9 & -8 & 1,5 & 360 & 4,5 & 700 & 11,5 & 1000 \\
      -9 & -12 & -0,8 & -7 & 1,6 & 380 & 4,6 & 700 & 12 & 1000 \\
      -8,5 & -12 & -0,7 & -6 & 1,7 & 380 & 4,7 & 700 & 12,5 & 1000 \\
      -8 & -12 & -0,65 & -5 & 1,8 & 390 & 4,8 & 700 & 13 & 1000 \\
      -7,5 & -12 & -0,6 & -4 & 1,9 & 410 & 4,9 & 700 & 14 & 1000 \\
      -7 & -11 & -0,55 & -2 & 2 & 440 & 5 & 710 & 15 & 1000 \\
      -6,5 & -11 & -0,5 & 0 & 2,1 & 440 & 5,2 & 720 & 16 & 1000 \\
      -6 & -11 & -0,45 & 1 & 2,2 & 460 & 5,4 & 720 & 17 & 1100 \\
      -5,5 & -10 & -0,4 & 8 & 2,3 & 470 & 5,6 & 720 & 18 & 1100 \\
      -5 & -10 & -0,35 & 17 & 2,4 & 480 & 5,8 & 760 & 19 & 1100 \\
      -4,5 & -10 & -0,3 & 32 & 2,5 & 500 & 6 & 780 &  &  \\
      \bottomrule
  \end{tabular}
\end{table}

\noindent Auch für die gelbe Spektrallinie soll nun die Grenzspannung bestimmt werden. Dafür wird in Abbildung \ref{fig:gelb}
die Wurzel des Stroms gegen die Spannung aufgetragen. Die ersten 7 Messungen mit positiven Photoströmen werden linearisiert.
Dabei ergeben sich als Fit-Parameter
\begin{align*}
  a_\text{gelb}&=\qty{27.4(0.9)}{\sqrt{\pico\ampere}\per\volt}\\
  b_\text{gelb}&=\qty{13.6(0.33)}{\sqrt{\pico\ampere}}.\\
\end{align*}
Die Grenzspannung ergibt sich dann durch \eqref{eq:Grenz} zu
\begin{equation*}
  U_\text{gelb}=\qty{-0.498(0.020)}{\volt}.
\end{equation*}
\begin{figure}[H]
  \centering
  \includegraphics{content/gelb.pdf}
  \caption{Messung der Stromabhängigkeit bei Variation der Brems-/Beschleunigungsspannung der gelben Spektrallinie im Grenzbereich}
  \label{fig:gelb}
\end{figure}
\begin{figure}[H]
  \centering
  \includegraphics{content/Allgelb.pdf}
  \caption{Messung der Stromabhängigkeit bei Variation der Brems-/Beschleunigungsspannung der gelben Spektrallinie im gesamten 20V Bereich}
  \label{fig:Allgelb}
\end{figure}
\subsection{Bestimmung der Grenzspannung und $h/e_0$}
Die zuvor berechneten Grenzspannungen bedeuten, dass selbst die energiereichsten Teilchen
die Anode nicht mehr erreichen. Daher kann $U_\text{Grenz} e_0$ als kinetische Energie der Elektronen 
interpretiert werden, die von den Photonen übertragen wird. Da die Photonen Energie mit $h$ proportional
zur Frequenz der Spektrallinien ansteigt, muss auch die Grenzspannung proportional zur Frequenz ansteigen.
Die Frequenz der einzelnen Spektrallinien ergibt sich durch $\nu=c/\lambda$.
Wenn nun die die Grenzspannungen der einzelnen Spektrallinien gegen die Frequenz aufgetragen werden, wie in Abbildung \ref{fig:Grenz}
dargestellt, lässt sich der lineare Verlauf erkennen. Die Steigung $a$ entspricht dann dem Verhältnis $h/e$ und der
Y-Achsenabschnitt $b$ lässt sich mit $e$ multipliziert in die Austrittsarbeit umrechnen.
Da die Grenzspannung der roten Spektrallinie im Vergleich zum Rest eindeutig nicht zum Verlauf passt, wird diese bei der linearen Regression nicht
berücksichtigt.
Die Fit-Parameter ergeben sich dann zu
\begin{align*}
  a&=\qty{-3.42(0.008)e-15}{\volt\second}\\
  b&=\qty{1.28(0.004)}{\volt}.
\end{align*}
Damit ergibt sich experimentell für $h/e_0$ und die Austrittsarbeit $W_A$
\begin{align}
  \frac{h}{e_0}&=\qty{-3.42(0.008)e-15}{\volt\second}\\
  W_A&=\qty{1.28(0.004)}{\eV}.
  \label{eq:ergebnis}
\end{align}
\begin{figure}[H]
  \centering
  \includegraphics{content/Grenzspannung.pdf}
  \caption{Grenzspannungen der verschiedenen Spektrallinien}
  \label{fig:Grenz}
\end{figure}