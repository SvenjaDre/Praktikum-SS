\section{Diskussion}
\label{sec:Diskussion}

Aufgrund dessen, dass der Einfallswinkel nicht mit den Reflexionswinkel übereinstimmen,
konnte das Reflexionsgesetz nicht bewiesen werden. 
Dies könnte unteranderem an Ablesungsfehler liegen.
Der Winkel konnte auf $\pm \qty{1}{°}$ gemessene werden. Jedoch auch mit dieser Abweichung
stimmen $\alpha_1$ mit $\alpha_2$ zum größten Teil nicht überein.\\


\noindent Der Theoriewert des Brechungsindex von Plexiglas liegt bei
\begin{equation*}
    n_\text{theorie} = 1.49.
\end{equation*}
Der im Versuch berechnete Brechungsindex beträgt
\begin{equation*}
    n_\text{berechnet} = 1.43.
\end{equation*}
Somit ergibt sich eine Abweichung von $\qty{4.02}{\%}$. \\

\noindent Die Abweichung des gemessen Ausfallswinkel $\beta$ und des berechneten $\beta_\text{berechnet}$ 
lassen sich aufgrund von Ablesungsfehler erklären. 
Die größte Abweichung hat der Einfallswinkel $\alpha = \qty{70}{°}$ mit $\qty{15.29}{\%}$.
Bei diesem Einfallswinkel haben somit auch die beiden berechneten Strahlenversätze die größte Abweichung von 
$\qty{13.86}{\%}$.
Die kleinste Abweichung der Strahlenversätze ist bei einem Einfallswinkel von $\alpha = \qty{10}{°}$ mit einer Abweichung von $\qty{3.22}{\%}$.
Durch das berechnen der Ausfallswinkel, kann der Strahlenversatz genauer berechnet werde, da Unsicherheiten nur beim ablesen 
der gemessenen Einfallswinkel auftreten können.
Bei der 1. Methode gibt es durch das Ablesen beider Winkel eine weitere Fehlerquelle.\\

\noindent Die Wellenlängen der Laser sind in der Theorie
\begin{equation*}
    \lambda_\text{rot,theorie} = \qty{635}{nm}
\end{equation*}
und
\begin{equation*}
    \lambda_\text{grün,theorie} = \qty{532}{nm}.
\end{equation*}
Die berechneten Wellenlängen sind
\begin{equation*}
    \lambda_\text{rot} = 606.82
  \end{equation*}
  und 
  \begin{equation*}
    \lambda_\text{grün} = 510.15
  \end{equation*}
  Damit besitzt der rote Laser eine Abweichung von $\qty{4.43}{\%}$ und der grüne 
  Laser eine Abweichung von $\qty{4.10}{\%}$.