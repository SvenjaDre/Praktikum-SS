\section{Auswertung}
\label{sec:Auswertung}

\subsection{Das Reflexionsgestz}
\label{sec:Reflexionsgesetz}

Die gemessenen Einfallswinkel $\alpha_1$ und Ausfallswinkel $\alpha_2$ werden in der Tabelle \ref{tab:Ref} dargestellt.

\begin{table}[H]
  \centering
  \caption{Messwerte zur Überprüfung des Reflexionsgesetz}
  \label{tab:Ref}
  \begin{tabular}{c c}
    \toprule
    $\alpha_1$ & $\alpha_2$ \\
    \midrule
    20 & 18 \\
    25 & 24 \\
    30 & 28 \\
    35 & 32 \\
    40 & 36 \\
    45 & 41 \\
    50 & 45 \\
    \bottomrule
  \end{tabular}
\end{table}
    
\subsection{Das Brechungsgesetz}
\label{sec:Brechungsgesetz}

Die gemessenen Einfallswinkel $\alpha$ und Brechungswinkel $\beta$ werden in der Tabelle \ref{tab:Bre} aufgelistet.
Zur Berechnung des Brechungsindex n von Plexiglas wird die Formel

\begin{equation}
  n = \frac{\sin(\alpha)}{\sin(\beta)}.
\end{equation}
Die gemessene Werte werden zur berechnung eingesetz und ebenfalls in der Tabelle \ref{tab:Bre} dargestellt.

\begin{table}[H]
  \centering
  \caption{Messwerte zur Überprüfung des Brechungsgesetz}
  \label{tab:Bre}
  \begin{tabular}{c c c}
    \toprule
    $\alpha$ & $\beta$ & $n$\\
    \midrule
    10 & 7    & \\
    20 & 13.5 & \\
    30 & 19.5 & \\
    40 & 25.5 & \\
    50 & 31   & \\
    60 & 36   & \\
    70 & 48.5 & \\
    \bottomrule
  \end{tabular}
\end{table}

