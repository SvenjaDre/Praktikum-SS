\section{Auswertung}
\label{sec:Auswertung}

\subsection{Das Reflexionsgestz}
\label{sec:Reflexionsgesetz}

Die gemessenen Einfallswinkel $\alpha_1$ und Ausfallswinkel $\alpha_2$ werden in der Tabelle \ref{tab:Ref} dargestellt.

\begin{table}[H]
  \centering
  \caption{Messwerte zur Überprüfung des Reflexionsgesetz.}
  \label{tab:Ref}
  \begin{tabular}{c c}
    \toprule
    $\alpha_1 /°$ & $\alpha_2 /°$ \\
    \midrule
    20 & 18 \\
    25 & 24 \\
    30 & 28 \\
    35 & 32 \\
    40 & 36 \\
    45 & 41 \\
    50 & 45 \\
    \bottomrule
  \end{tabular}
\end{table}

\subsection{Das Brechungsgesetz}
\label{sec:Brechungsgesetz}

Die gemessenen Einfallswinkel $\alpha$ und Brechungswinkel $\beta$ werden in der Tabelle \ref{tab:Bre} aufgelistet.
Zur Berechnung des Brechungsindex n von Plexiglas wird die Formel

\begin{equation}
  \label{eq:B}
  n = \frac{\sin(\alpha)}{\sin(\beta)}.
\end{equation}
Die gemessene Werte werden zur Berechnung eingesetz und ebenfalls in der Tabelle \ref{tab:Bre} dargestellt.

\begin{table}[H]
  \centering
  \caption{Messwerte zur Überprüfung des Brechungsgesetz.}
  \label{tab:Bre}
  \begin{tabular}{c c c}
    \toprule
    $\alpha /°$ & $\beta /°$ & $n$\\
    \midrule
    10 & 7    & 1.42 \\
    20 & 13.5 & 1.46\\
    30 & 19.5 & 1,49\\
    40 & 25.5 & 1.49\\
    50 & 31   & 1.48\\
    60 & 36   & 1.47\\
    70 & 48.5 & 1.25\\
    \bottomrule
  \end{tabular}
\end{table}

\noindent Für die Berechnung des Brechungsindex für Plexiglas wird der Mittelwert berechnet.
Daraus ergibt sich ein Brechungsindex von
\begin{equation*}
  n = 1.43.
\end{equation*} 

\noindent Um die Lichtgeschwindigkeit v im Plexiglas zu bestimmen wird der Zusammenhang \ref{eq:Brechung} verwendet.
Dabei ist $n_1$ der Brechungsindex von Luft, welcher $n_1 = 1$ beträgt und $v_1 = 2.99*10^8$.
Somit ergibt sich für die Lichtgeschwindigkeit im Plexiglas 
\begin{equation*}
  v_\text{Plexi} = \qty{208601391.65}{m/s} .
\end{equation*}

\subsection{Planparallele Platte}
Zur Bestimmung des Strahlenversatz s werden zwei Methoden verwendet.
Bei der ersten Methode werden die gemessenen Einfalls- und Ausfallswinkel aus der Tabelle \ref{tab:Bre} verwendet und 
in die Formel \ref{eq:Strahlenversatz} eingesetzt.
Dabei beträgt die Dicke der Platte $d = \qty{5.85}{cm}$

Bei der zweiten Methode wird der Winkel $\beta_\text{berechnet}$ mithilfe von \ref{eq:B} berechnet.
Anschließend wird der Strahlenversatz bestimmt.
Für den Brechungsindex wird das Ergebnis $n = 1.43$ aus \ref{sec:Brechungsgesetz} verwendet.

Die Ergebnisse werden in der Tabelle \ref{tab:Plan} dargestellt.
\begin{table}[H]
  \centering
  \caption{Berechneter Strahlenversatz mithilfe zwei verschiedener Methoden.}
  \label{tab:Plan}
  \begin{tabular}{c c c c c}
    \toprule
  & \multicolumn{2}{c}{1.Methode} & \multicolumn{2}{c}{2.Methode} \\
\cmidrule(lr){2-3}\cmidrule(lr){4-5}
    $\alpha / °$ & $\beta  /° $ & $s /cm$ &  $\beta_\text{berechnet}  /°$ & $s /cm$ \\
    \midrule
    10 & 7    & 0.30 & 6.97 & 0.31\\  
    20 & 13.5 & 0.68 & 13.83 & 0.64\\    
    30 & 19.5 & 1.13 & 20.46 & 1.03\\  
    40 & 25.5 & 1.62 & 26.71 & 1.50\\  
    50 & 31   & 2.22 & 32.39 & 2.09\\  
    60 & 36   & 2.94 & 37.27 & 2.84\\  
    70 & 48.5 & 3.23 & 41.08 & 3.75\\  
    \bottomrule
  \end{tabular}
\end{table}

\subsection{Prisma}

Bei dem verwendeten Prisma handelt es sich um ein Prisma aus Kornglas.
Dieser hat einen brechenden Winkel von $\gamma = \qty{60}{°}$.
Um die Ablenkung $\delta$ der beiden Laser am Prisma zu berechnen, wird die Formel \ref{eq:Prisma} verwendet.
Die berechneten Ergebnisse werden in der Tabelle \ref{tab:Prisma} dargestellt.

\begin{table}[H]
  \centering
  \caption{Messwerte und berechnete Ablenkung zweier Laser an einem Prisma.}
  \label{tab:Prisma}
  \begin{tabular}{c c c c c}
    \toprule
  & \multicolumn{2}{c}{Grüner Laser} & \multicolumn{2}{c}{Roter Laser} \\
\cmidrule(lr){2-3}\cmidrule(lr){4-5}
    $\alpha_1 / °$ & $\alpha_2  /° $ & $\delta_\text{grün}$ &  $\alpha_2  /°$ & $\delta_\text{rot}$ \\
    \midrule
    30 & 74    & 44   & 75 & 45 \\  
    35 & 65    & 40   & 66 & 41 \\    
    40 & 59.5  & 39.5 & 58 & 38 \\  
    45 & 52    & 37   & 52 & 37 \\  
    50 & 47    & 37   & 46 & 36 \\   
    \bottomrule
  \end{tabular}
\end{table}


\subsection{Beugung am Gitter}

