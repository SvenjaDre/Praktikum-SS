\section{Zielsetzung}
In diesem Versuch wird die Halbwertszeit und die Zerfallskurve eines Isotops und eines istabilen Isotopengemische bestimmt.
Zur Bestimmung wird die Aktivierung mit Neutronen verwendet.
\section{Theorie}
\label{sec:Theorie}
Die Halbwertszeit $T$ beschreibt die Wahrscheinlichkeit des Zerfalls eines Nuklids
Um die Halbwertszeit zu bestimmen werden Nuklide verwendet, dessen Halbwertszeit zwischen Sekunden und Stunden messbar sind.
Diese Nuklide entstehen, wenn ein stabiler Kern mit Neutronen beschossen wird.

\subsection{Kernreaktionen mit Neutronen}
Wird ein Kern mit einem Neutronen beschossen, wird dies vom Kern absorpiert.
Es entsteht ein neuer Kern A* der als Zwischkern oder Compoundkern bezeichnet wird.
Die aufgenommene Energie verteilt sich auf die Nukleonen.
Aufgrund dessen kommt es zur sogenannten Aufheizung des Zwischenkerns, da die Nukleonen einen höheren Energiezustand befinden.
Damit der der Zwischenkern wieder in den Grundzustand gelangt, werden $\gamma$-Quanten emittiert, da das aufgenommene Neutron nicht wieder abgestoßen werden kann.
\begin{equation}
    \ce{^{m}_{z}A} + \ce{^{1}_{0}n} -> \ce{^{m+1}_{z}A^*} -> \ce{^{m+1}_{z}A} + \gamma
\end{equation}

\cite{sample}
