\section{Diskussion}
Für eine sinnvolle Untersuchung des Zerfallsverhalten von Isotopen muss am
Ort des Versuchs die Hintergrundaktivität bestimmt werden. Diese ergibt sich zu 
$A_\text{Hint}=\qty{0.357(0.026)}{\per\s}$. Damit ist auch die geforderte maximale Unschärfe von 
10\% gegeben. Die Halbwertzeit von Vanadium ist experimentell zu $T_V=\qty{224(7)}{\second}$ bestimmt worden. Der Theoriewert weicht mit
$T_V=\qty{225}{\second}$ \cite{V52} um deutlich weniger als der angegebene Fehler und um weniger als 1\% von dem experimentell ermittelten ab. Für 
eine genauere Untersuchung müsste die Überbrückungszeit zwischen Aktivierung und Messung reduziert werden und mehrere Messreihen miteinander
verglichen werden. Der Literaturwert des langlebigen Isotops \isotope[108]{Ag} ist mit
$T_V=\qty{142}{\second}$ und der des kurzlebigen Isotops \isotope[110]{Ag} mit  $T_V=\qty{24.6}{\second}$ angegeben \cite{Ag108},\cite{Ag110}.
Nun wird zunächst der langlebige Zerfall betrachtet. In der ersten Messung wird die Halbwertszeit $T_\text{\isotope[108]{Ag}}=\qty{249(76)}{\second} $
und in der zweiten $T_\text{\isotope[108]{Ag}}=\qty{184(59)}{\second} $ gemessen. Was dabei auffällt, ist, dass 
beide Werte deutlich größer sind als der theoretisch vorhergesagte, wobei der Theoriewert in der zweiten Messung noch in der Fehlerbreite liegt. 
Dies ist allerdings nur dadurch bedingt, dass der sich aus der Ausgleichsrechnung ergebenden Fehler bei beiden Messungen sehr groß ist.
Dies hat hauptsächlich statistische Ursprünge. Da der Fehler sich aus der Wurzel der Zerfälle ergibt, werden die relativen Fehler, die 
in der logarithmischen Darstellung relevant sind, bei kleiner werdenden detektierten Zerfällen immer größer. Also die Messwerte streuen deutlich mehr.
Für die Bestimmung der Ausgleichsgerade werden aber nur die Messwerte berücksichtigt bei denen 
der kurzlebige keinen Einfluss mehr hat, wodurch die für die Ausgleichsrechnung herrangezogenen Werte 
keine allzu große Aussagekraft mehr besitzen. Daher eine gute Lösung das Experiment aussagekräftiger zu machen, wäre einfach eine massereichere 
Probe zu nehmen. Dadurch würden auch bei größeren Zeiten noch aussagekräftige Datenpunkte genommen werden. 
Als Resultat des kurzlebigen Isotops ergeben sich Halbwertszeiten von $T_\text{\isotope[110]{Ag}}=\qty{27.3(9.42)}{\second}$ und 
$T_\text{\isotope[110]{Ag}}=\qty{23.1(1.19)}{\second}$. Die Literaturwert liegt in der Fehlerbreite der ersten Messung. Generell ist der relative Fehler auch geringer. 
Die erste Messung weicht um 11\% und die zweite um 6\% vom Literaturwert ab. Dabei ist natürlich zu beachten, dass 
beide Halbwertszeiten aus fehlerhaft korrigierten Daten gewonnen worden ist. Es werden ja die vermuteten Zerfälle des langlebigen Zerfalls zunächst abgezogen.
Außerdem hat sich die Isotopenanzahl vermutlich schon zwischen der Entnahme und dem Messbeginn einmal halbiert, weswegen der kurzlebige Zerfall nur sehr kurzzeitig messbar
ist. Da zwei verschiedene Messreihen genommen worden sind um die gleiche physikalische Größe zu bestimmen bietet es sich an einen gewichteten Mittelwert zu bilden.
\begin{equation}
    T_\text{110,Mittel}=\frac{\frac{1}{\Delta T_\text{110,1}}^2\cdot T_\text{110,1}+\frac{1}{\Delta T_\text{110,2}}^2\cdot T_\text{110,2}}{\frac{1}{\Delta T_\text{110,1}}^2+\frac{1}{\Delta T_\text{110,1}}^2}
    =\frac{\frac{1}{\qty{9.42}{\second}}^2\cdot \qty{27.3}{\second}+\frac{1}{\qty{1.19}{\second}}^2\cdot \qty{23.1}{\second}}{\frac{1}{\qty{9.42}{\second}}^2+\frac{1}{\qty{1.19}{\second}}^2}=\qty{23.2}{\second}
\end{equation}
Der neue Fehler liegt dann bei
\begin{equation}
    \Delta T = \frac{\Delta T_\text{110,1} \cdot \Delta T_\text{110,2} }{\sqrt{\Delta T_\text{110,1}^2+\Delta T_\text{110,2}^2}}
    =\frac{\qty{9.42}{\second}\cdot \qty{1.19}{\second}}{\sqrt{(\qty{9.42}{\second})^2+(\qty{1.19}{\second})^2}}=\qty{1.18}{\second}
\end{equation}
Da die erste Messreihe keine wirklich genaue Aussage über die Halbwertzeit gibt, ist die Halbwertzeit auch nach Verrechnung kaum unterschiedlich zu dem Ergebnis aus Messung 2.
\label{sec:Diskussion}
