\section{Theorie}
\label{sec:Theorie}

Bei $\alpha$-Strahlung handelt es sich um einen Helium Kern, der durch radioaktiven Zerfall entsteht.
Durch Messung der Reichweite der Strahlung, kann deren Energie bestimmt werden.
Die $\alpha$-Strahlung verliert an Energie durch Ionisierungsprozessesen, Anregung und Dissoziation von Molekülen.
Zudem kann es zum Energieverlust kommen, wenn die Strahlung durch Materie läuft und die Teilchen mit der Materie elastisch zusammen stoßen.
Ein Teil der Energie wird abgegeben.
Dabei ist der Energieverlust abhängig von der Energie der $\alpha$-Strahlung, sowie die Dichte der Materie.
Der Energieverlust -$\frac{dE_\alpha}{dx}$ kann bei großen Energien mithilfe der Bethe-Bloch-Gleichung 
\begin{equation}
    -\frac{dE_\alpha}{dx} = \frac{z^2 e^4}{4 \pi \epsilon_0 m_e}
     \frac{n Z}{v^2}ln\biggl(\frac{2 m_e v^2}{I}\biggr)
\end{equation}
berechnet werden.
Z bechreibt dabei die Ordnungszahl, n die Teilchendichte, z die Ladung, I die Ionisierungsenergie des Targetgases und v die Geschwindigkeit der $\alpha$- Strahlung.
Die Bethe-Bloch-Gleichung kann jedoch dann nicht mehr verwendet werden, wenn es zuLadungsaustauschprozesse kommt. 
Diese treten dann auf,wenn sehr kleine Energien vorliegen.



\cite{sample}
