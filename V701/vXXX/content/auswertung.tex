\section{Auswertung}
\label{sec:Auswertung}

\subsection{Messung der Zählrate und der Energie von Alpha-Strahlung}
In Tabelle \ref{tab:Mess1} sind die gemessene Zählrate und der Channel bei variierendem Druck gemessen worden.
Dabei beträgt die Entfernung zwischen Detektor und Präparat $\qty{2.3}{\centi\meter}$. Eine Messvorgang dauert 2 Minuten.
Die Channel sind ein indirektes Maß für die Energie. Es wird angenommen, dass die Alpha-Teilchen bei
einem Druck von $\qty{0}{\milli\bar}$ eine maximale Energie von $\qty{4}{\mega\electronvolt}$ besitzen.
Damit lässt sich bei der Messung aus Tabelle \ref{tab:Mess1} der Channel 816 der Energie $\qty{4}{\mega\electronvolt}$
zuordnen. Die anderen Channel folgen dieser Zuordnung linear. Also Channel 408 entspricht
einer Energie von $\qty{2}{\mega\electronvolt}$.
Damit ergibt sich für die Energie
\begin{equation}
  E=\frac{\text{Channel}}{816}\cdot \qty{4}{\mega\electronvolt}.
  \label{eq:En1}
\end{equation}
Genauso lässt sich mit Gleichung \eqref{eq:ptox} dem Druck eine effektive Entfernung zum Präperat
bei Normaldruck bestimmen. Dabei ist $x_0=\qty{2.3}{\centi\meter}$.
Die korrespondierenden Längen und Entfernungen sind ebenfalls in Tabelle \ref{tab:Mess1} angegeben.
\begin{table}[H]
  \centering
  \caption{Messreihe 1 bei einer Entfernung von $\qty{2.3}{\centi\meter}$ und nach 2 Minuten}
  \label{tab:Mess1}
  \begin{tabular}{S[table-format=4] S[table-format=3] S[table-format=5] S[table-format=1.2] S[table-format=1.2]}
    \toprule
      {Druck/mbar} & {Channel} & {Zählrate/2min}&{effektive Entfernung/$\unit{\centi\meter}$} &{Energie in MeV}\\
    \midrule
    0 & 816 & 84290 & 0.00 & 4.00 \\
    50 & 768 & 83234 & 0.11 & 3.76 \\
    100 & 751 & 82737 & 0.23 & 3.68 \\
    150 & 731 & 82099 & 0.34 & 3.58 \\
    200 & 711 & 82005 & 0.45 & 3.49 \\
    250 & 679 & 80538 & 0.57 & 3.33 \\
    300 & 640 & 80144 & 0.68 & 3.14 \\
    350 & 632 & 79189 & 0.80 & 3.10 \\
    400 & 647 & 78107 & 0.91 & 3.17 \\
    450 & 608 & 77703 & 1.02 & 2.98 \\
    500 & 563 & 76030 & 1.14 & 2.76 \\
    550 & 542 & 75554 & 1.25 & 2.66 \\
    600 & 511 & 74284 & 1.36 & 2.50 \\
    650 & 476 & 73075 & 1.48 & 2.33 \\
    700 & 448 & 71075 & 1.59 & 2.20 \\
    750 & 416 & 69127 & 1.71 & 2.04 \\
    800 & 399 & 66378 & 1.82 & 1.96 \\
    850 & 352 & 60823 & 1.93 & 1.73 \\
    900 & 295 & 49233 & 2.05 & 1.45 \\
    950 & 286 & 41287 & 2.16 & 1.40 \\
    1000 & 271 & 26338 & 2.27 & 1.33 \\
      \bottomrule
  \end{tabular}
\end{table}
\noindent Zur besseren Einordnung wird noch eine zweite Messreihe als Vergleich aufgenommen.
Die Ergebnisse sind in Tabelle \ref{tab:Mess2} aufgeführt. Die zweite Messreihe ist bei einer größeren
Distanz zwischen Detektor und Präparat von $\qty{3.2}{\centi\meter}$ gemessen worden.
In dieser Messreihe sind im Vakuum die meisten Detektionen bei Channel 815 aufgetreten.
Damit ergeben sich die dazugehörigen Energien durch
\begin{equation}
  E=\frac{\text{Channel}}{815}\cdot \qty{4}{\mega\electronvolt}.
  \label{eq:En2}
\end{equation} 
Die effektiven Entfernungen ergeben sich bei der 2. Messreihe erneut nach Gleichung \eqref{eq:ptox}.
Die Normaldruckentfernung $x_0$ beträgt dabei $\qty{3.2}{\cent\meter}$.
\begin{table}[H]
  \centering
  \caption{Messreihe 2 bei einer Entfernung von $\qty{3.2}{\centi\meter}$ und nach 2 Minuten}
  \label{tab:Mess2}
  \begin{tabular}{S[table-format=4] S[table-format=3] S[table-format=5] S[table-format=1.2] S[table-format=1.2]}
    \toprule
      {Druck/mbar} & {Channel} & {Zählrate}&{effektive Entfernung/$\unit{\centi\meter}$} &{Energie in MeV}\\
    \midrule
          0 & 815 & 54534 & 0.00 & 4.00 \\
      50 & 768 & 54488 & 0.16 & 3.77 \\
      100 & 736 & 53859 & 0.32 & 3.61 \\
      150 & 704 & 52402 & 0.47 & 3.46 \\
      200 & 687 & 52286 & 0.63 & 3.37 \\
      250 & 643 & 51419 & 0.79 & 3.16 \\
      300 & 591 & 50703 & 0.95 & 2.90 \\
      350 & 576 & 49985 & 1.11 & 2.83 \\
      400 & 560 & 49203 & 1.26 & 2.75 \\
      450 & 531 & 48215 & 1.42 & 2.61 \\
      500 & 480 & 46999 & 1.58 & 2.36 \\
      550 & 439 & 45683 & 1.74 & 2.15 \\
      600 & 399 & 43407 & 1.89 & 1.96 \\
      650 & 371 & 40089 & 2.05 & 1.82 \\
      700 & 300 & 33582 & 2.21 & 1.47 \\
      750 & 276 & 15318 & 2.37 & 1.35 \\
      800 & 267 & 5750 & 2.53 & 1.31 \\
      850 & 267 & 501 & 2.69 & 1.31 \\
      \bottomrule
  \end{tabular}
\end{table}
\subsubsection{Energieverlust von Alpha-Strahlung}

\begin{figure}[H]
  \centering
  \includegraphics{Energiemaxima1.pdf}
  \caption{Energie von Alpha-Strahlung bei $\qty{2.3}{\centi\meter}$ Entfernung}
  \label{fig:Energie1}
\end{figure}

\begin{figure}[H]
  \centering
  \includegraphics{Energiemaxima2.pdf}
  \caption{Energie von Alpha-Strahlung bei $\qty{3.2}{\centi\meter}$ Entfernung}
  \label{fig:Energie2}
\end{figure}

\subsubsection{Bestimmung der mittleren Reichweite von Alpha-Strahlung in Luft}

\begin{figure}[H]
  \centering
  \includegraphics{Zählrate1.pdf}
  \caption{Detektierte Zerfälle pro 2 Minuten bei $\qty{2.3}{\centi\meter}$ Entfernung}
  \label{fig:Energie1}
\end{figure}

\begin{figure}[H]
  \centering
  \includegraphics{Zählrate2.pdf}
  \caption{Detektierte Zerfälle pro 2 Minuten bei $\qty{2.3}{\centi\meter}$ Entfernung}
  \label{fig:Energie1}
\end{figure}

\subsection{Statistik des radioaktiven Zerfalls}

\begin{table}[H]
  \centering
  \caption{Messung der Zählrate in $\qty{3.2}{\centi\meter}$ Entfernung nach 10s}
  \label{tab:Mess3}
  \begin{tabular}{S[table-format=4] S[table-format=4]S[table-format=4] S[table-format=4]}
    \toprule
     \multicolumn{4}{c} {Zählrate}\\
    \midrule
    4347    & 4174 & 4087 & 4305 \\
    4536    & 4103 & 4124 & 4316 \\
    4329    & 4333 & 4040 & 3963 \\
    4261    & 4187 & 4065 & 4212 \\
    4127    & 4342 & 4026 & 4136 \\
    4366    & 4288 & 4026 & 4438 \\
    4459    & 4350 & 4195 & 4342 \\
    4342    & 4312 & 4433 & 3976 \\
    4188    & 4151 & 4016 & 4027 \\
    4532    & 4297 & 4271 & 4329 \\
    4445    & 4323 & 4056 & 4195 \\
    3971    & 4179 & 4371 & 4261 \\
    4270    & 4294 & 4090 & 4120 \\
    4065    & 4019 & 4250 & 4250 \\
    3985    & 4346 & 4186 & 4041 \\
    4434    & 4019 & 4125 & 4245 \\
    4039    & 4388 & 4382 & 4165 \\
    4085    & 4167 & 4073 & 4259 \\
    4176    & 4159 & 4224 & 4361 \\
    4405    & 4389 & 4980 & 4257 \\
    4046    & 4232 & 4279 & 4258 \\
    4272    & 4294 & 4082 & 4190 \\
    4300    & 4263 & 4023 & 4184 \\
    4111    & 3956 & 4302 & 4146 \\
    4284    & 4229 & 4422 & 4309 \\
    \bottomrule
  \end{tabular}
\end{table}

\begin{figure}[H]
  \centering
  \includegraphics{Statistik.pdf}
  \caption{Detektierte Zerfälle in 10s bei $\qty{3.2}{\centi\meter}$ Entfernung im Vergleich mit einer Normalverteilung}
  \label{fig:Energie1}
\end{figure}

\begin{figure}[H]
  \centering
  \includegraphics{Statistik1.pdf}
  \caption{Detektierte Zerfälle in 10s bei $\qty{3.2}{\centi\meter}$ Entfernung im Vergleich mit einer Poissonverteilung}
  \label{fig:Energie1}
\end{figure}