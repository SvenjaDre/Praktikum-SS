\section{Diskussion}
\label{sec:Diskussion}
Die Ergebnisse bei einer Entfernung von $\qty{2.3}{cm}$ lauten:
\begin{equation*}
    \frac{\symup{d}E}{\symup{d}x}=\qty{-1.17}{\mega\electronvolt\per\centi\meter}
\end{equation*}

\begin{equation*}
   x = \qty{2.13 \pm 0.15}{cm}
\end{equation*}
   
\begin{equation*}
    E = \qty{1.53 \pm 0.18}{MeV}
\end{equation*}
Und die Ergebnisse bei einer Entfernung von $\qty{3.2}{cm}$ sind:
\begin{equation*}
    \frac{\symup{d}E}{\symup{d}x}=\qty{-1.05}{\mega\electronvolt\per\centi\meter}
\end{equation*}

\begin{equation*}
   x = \qty{2.25 \pm 0.37}{cm}
\end{equation*}
   
\begin{equation*}
    E = \qty{1.61 \pm 0.39}{MeV}
\end{equation*}
\\


\noindent Die Abweichungen könnten dadurch entstanden sein, dass 
der Druck manuel über das Belüftungsventil eingestellt wurde.
Aufgrund dessen das diese per Hnad eingestellt wurde, konnte es zu ungenauigkeiten gekommen sein. 
Weitere Ungenauigkeiten könnten durch das ablesen des Abstandes zwischen Detektor und Am-Präparat.\\


\noindent Über die Statsitische Verteilung des radioaktiven Zerfalls lässt sich sagen, dass 
die Normalverteilung am besten an die gemessene Verteilung annährt.
Der Verlauf der Normalkurve verläuft nahe der im Histogramm aufgetragenen Verteilung.
Während der Verlauf Poissonverteilung deutlich schmaler verläuft und somit innerhalb der Balken des Histogramms verläuft.
Die Balken des Histogramms befinden sich bei der Normalverteilung zum größten Teil unterhalb der Kurve.
Zusammenfassen lässt sich sagen, dass zur Beschreibung der Statistik sich die Normalverteilung besser eignet.
